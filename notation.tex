\section*{Notation and conventions}\label{sec:notation}

We usually follow the notation of \cite{hit87a}:
\begin{itemize}
    \item $M$: compact Riemann surface of genus $g$.
    \item $O\to M$: trivial line bundle.
    \item $K\to M$: canonical line bundle.
    \item More generally, $O_{X}$ and $K_{X}$ denote the trivial and canonical line bundles over a complex manifold $X$.
\end{itemize}
Every now and then we will also use some other standard notation, for instance:
\begin{itemize}
    \item For a vector bundle $E\to M$ we denote $\mu(E):=\deg{E}/\rk{E}$.
    \item Let $X$ be a complex manifold and $E\to X$ a (holomorphic/algebraic) vector bundle.
	Then we denote by $\E$ its sheaf of sections.
	The assignement $E\mapsto \E$ defines an equivalence of categories between vector bundles on $X$ and locally free sheaves of $\O_{X}$-modules of finite rank, which we will refer to simply as locally free sheaves.
	We can recover $E$ from $\E$ either using cocycles \cite[Lemma 4.8]{voi02} or by using the relative spectrum \cite[Exercise II.5.18]{har77}.
	Following the second approach we would write
	\[ E=\Spec_{X}(S(\E\dual))=:\V(\E\dual), \]
	where $S(-)$ denotes the symmetric algebra.
    \item $\O$ and $\w$ denote the trivial and canonical invertible sheaves on $M$.
	More generally, $\O_{X}$ and $\w_{X}$ denote the trivial and canonical invertible sheaves on a complex manifold $X$,
    \item Let $E$ be again a vector bundle on a complex manifold $X$.
	We will denote its projectivisation by $\P(E)$, which is obtained form $E$ without its zero section by quotienting out the fibre-wise multiplication by scalars.
	Taking again the algebraic approach we can write
	\[ \P(E)=\Proj_{X}(S(\E\dual))=:\P(\E\dual). \]
\end{itemize}

\bibliographystyle{alpha}
\bibliography{main}
\vfill

\end{document}
