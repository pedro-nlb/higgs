\documentclass[A4paper, 12pt, british, reqno]{amsart}

%%%%%%%%%% CONTENTS OF PREAMBLE %%%%%%%%%%

% Packages ----------------- Line 10
% General things ----------- Line 72
% Font definitions --------- Line 83
% Theorem environments ----- Line 228
% Tikzcd ------------------- Line 400
% Author, title, etc ------- Line 420

%%%%%%%%%% PACKAGES %%%%%%%%%%

\usepackage[T1]{fontenc}
% This uses 8-bit font encoding (with 256 glyphs) instead of the default 7-bit font encoding (with 128 glyphs). For example, with this option ö is a single glyph in the font, whereas on the 7-bit font encoding the font ö is made by adding an accent to the existing glyph o. A bad consequence of not using this package is that you cannot properly copy-paste such words form the output pdf file. Also, for some reason, funny stuff happens with |, < and > in text.

% Some people suggest to load fontenc before inputenc, most agree that it does not matter.

\usepackage[utf8]{inputenc}
% When you type ä in an editor set up for utf8, the machine stores the character number 228. When TeX reads the file it finds the character number 228 and the macros of inputenc transform this into \"a. Finally fontenc does its thing and transforms this into the command print character 228 (otherwise the two things would be printed separatedly as explained in fontenc).

\usepackage[UKenglish]{babel}
% To manage culturally determined typographical and similar rules, in this case for british english. Some people suggest to load babel after fontenc to avoid warnings, although most agree that it does not matter.

\usepackage{mathtools}
% Loads the amsmath package (\usepackage{amsmath}: miscellaneous improvements such as the commands \DeclareMathOperator and \text). It fixes some quirks it has and adds some useful settings, symbols and environments. It improves the aesthetics as well.

\usepackage{amssymb}
% Extended symbol collection, e.g. \Cap and \Cup. More importantly: the \mathbb command! It loads the amsfonts package (\usepackage{amsfonts}: fraktur letters, bold Greek letters...), so we do not need to include it in the preamble anymore.

\usepackage{mathrsfs}
% Font package (only supports upper case letters).

\usepackage{enumitem}
% To control the layout of enumerate, itemize and description. It supersedes the enumerate package.

\usepackage{tikz-cd}
% To draw commutative diagrams.
\usetikzlibrary{decorations.markings}
% For open and closed immersions.

\usepackage{graphicx}
% An extension of the graphics package, with optional arguments for the \includegraphics command.

\usepackage{todonotes}
% To write to do notes use the command \todo.

\usepackage{xcolor}
% To write in colors.

\usepackage{mathdots}
% To draw diagonal dots.

\usepackage{marginnote}
% To write on margins.

\usepackage{manfnt}
% To draw dangerous bent symbol.

\usepackage{libertine,libertinust1math}

\usepackage[mathscr]{euscript}
% For mathscr font family.

\usepackage{float}
% Improved interface for floating objects such as figures and tables, introducing for example the H modifier to force the position of a float in the page or the boxed float. Should be loaded before hyperref.

\usepackage[backref=page]{hyperref}
% To handle cross-referencing and produce hypertext links in the document. It should be loaded last (with few exceptions), because it redefines many LaTeX commands.
% The backref option inserts links on each bibliography entry to the pages in which the citation was used.
%% The hidelinks option removes colors and boxes around links, but the links remain clickable. On firefox the links are even highlighted when the mouse pointer passes over them.
\renewcommand{\backref}[1]{$\uparrow$~#1}
% Adds an upwards arrow before referencing to the pages in which the citations appear.

\usepackage[noabbrev]{cleveref}
% Enhances cross-referencing features, e.g. to reference to a theorem and automatically include the word theorem.
% No abbreviature option to write figure instead of fig. etc.

%%%%%%%%%% GENERAL THINGS %%%%%%%%%%

% Limit table of contents to section titles
\setcounter{tocdepth}{1}

% Sloppy formatting -- often looks better
\sloppy

% To make long aligned equations look better
\allowdisplaybreaks

%%%%%%%%%% FONT DEFINITIONS %%%%%%%%%%

% Sheaves
\newcommand{\sA}{\mathscr{A}}
\newcommand{\sB}{\mathscr{B}}
\newcommand{\sC}{\mathscr{C}}
\newcommand{\sD}{\mathscr{D}}
\newcommand{\E}{\mathscr{E}} % Exception (Vector bundles)
\newcommand{\F}{\mathscr{F}} % Exception (Coherent sheaves)
\newcommand{\G}{\mathscr{G}} % Exception (Coherent sheaves)
\newcommand{\sH}{\mathscr{H}}
\newcommand{\sHom}{\mathscr{H}\hspace{-0.2mm} om} % Exception (Hom-sheaf)
\newcommand{\sEnd}{\mathscr{E}\hspace{-0.2mm} nd} % Exception (End-sheaf)
\newcommand{\I}{\mathscr{I}} % Exception (Ideal sheaves)
\newcommand{\sJ}{\mathscr{J}}
\newcommand{\sK}{\mathscr{K}}
\renewcommand{\L}{\mathscr{L}} % Exception (Line bundles)
\newcommand{\M}{\mathscr{M}} % Exception (Line bundles)
\newcommand{\sN}{\mathscr{N}}
\renewcommand{\O}{\mathscr{O}} % Exception (Structure sheaf)
\newcommand{\sP}{\mathscr{P}}
\newcommand{\sQ}{\mathscr{Q}}
\newcommand{\sR}{\mathscr{R}}
\newcommand{\sS}{\mathscr{S}}
\newcommand{\sT}{\mathscr{T}}
\newcommand{\sU}{\mathscr{U}}
\newcommand{\sV}{\mathscr{V}}
\newcommand{\sW}{\mathscr{W}}
\newcommand{\w}{\omega} % Addition (Canonical sheaf)
\newcommand{\sX}{\mathscr{X}}
\newcommand{\sY}{\mathscr{Y}}
\newcommand{\sZ}{\mathscr{Z}}

% Mathcal fonts
\newcommand{\calA}{\mathcal{A}}
\newcommand{\calB}{\mathcal{B}}
\newcommand{\calC}{\mathcal{C}}
\newcommand{\calD}{\mathcal{D}}
\newcommand{\calE}{\mathcal{E}}
\newcommand{\calF}{\mathcal{F}}
\newcommand{\calG}{\mathcal{G}}
\newcommand{\calH}{\mathcal{H}}
\newcommand{\calI}{\mathcal{I}}
\newcommand{\calJ}{\mathcal{J}}
\newcommand{\calK}{\mathcal{K}}
\newcommand{\calL}{\mathcal{L}}
\newcommand{\calM}{\mathcal{M}}
\newcommand{\calN}{\mathcal{N}}
\newcommand{\calO}{\mathcal{O}}
\newcommand{\calP}{\mathcal{P}}
\newcommand{\calQ}{\mathcal{Q}}
\newcommand{\calR}{\mathcal{R}}
\newcommand{\calS}{\mathcal{S}}
\newcommand{\calT}{\mathcal{T}}
\newcommand{\U}{\mathcal{U}} % Exception (Open covers)
\newcommand{\calV}{\mathcal{V}}
\newcommand{\calW}{\mathcal{W}}
\newcommand{\X}{\mathcal{X}} % Exception (Families of varieties)
\newcommand{\Y}{\mathcal{Y}} % Exception (Families of varieties)
\newcommand{\calZ}{\mathcal{Z}}

% Blackboard Bold Symbols
\newcommand{\A}{\mathbb{A}} % Exception (Affine space)
\newcommand{\bbB}{\mathbb{B}}
\newcommand{\C}{\mathbb{C}} % Exception (Complex numbers)
\newcommand{\bbD}{\mathbb{D}}
\newcommand{\bbE}{\mathbb{E}}
\newcommand{\bbF}{\mathbb{F}}
\newcommand{\bbG}{\mathbb{G}}
\newcommand{\Gm}{\mathbb{G}_{\mathrm{m}}} % Addition (Punctured affine line)
\newcommand{\bbH}{\mathbb{H}}
\newcommand{\bbI}{\mathbb{I}}
\newcommand{\bbJ}{\mathbb{J}}
\newcommand{\bbK}{\mathbb{K}}
\renewcommand{\k}{\Bbbk} % Addition (Field)
\newcommand{\bbL}{\mathbb{L}}
\newcommand{\bbM}{\mathbb{M}}
\newcommand{\N}{\mathbb{N}} % Exception (Natural numbers)
\newcommand{\bbO}{\mathbb{O}}
\renewcommand{\P}{\mathbb{P}} % Exception (Projective space)
\newcommand{\Q}{\mathbb{Q}} % Exception (Rational numbers)
\newcommand{\R}{\mathbb{R}} % Exception (Real numbers)
\newcommand{\bbS}{\mathbb{S}}
\newcommand{\bbT}{\mathbb{T}}
\newcommand{\bbU}{\mathbb{U}}
\newcommand{\V}{\mathbb{V}} % Exception (Geometric vector bundle)
\newcommand{\bbW}{\mathbb{W}}
\newcommand{\bbX}{\mathbb{X}}
\newcommand{\bbY}{\mathbb{Y}}
\newcommand{\Z}{\mathbb{Z}} % Exception (Integers)

% Boldfont (categories)
\newcommand{\bfA}{\mathbf{A}}
\newcommand{\Ab}{\mathbf{Ab}}
\newcommand{\bfB}{\mathbf{B}}
\newcommand{\bfC}{\mathbf{C}}
\newcommand{\Cat}{\mathbf{Cat}} % Addition (Categories)
\newcommand{\Coh}{\mathbf{Coh}} % Addition (Coherent sheaves)
\newcommand{\D}{\mathbf{D}} % Exception (Derived category)
\newcommand{\Db}{\mathbf{D}^{\mathrm{b}}} % Addition (Bounded derived category)
\newcommand{\bfE}{\mathbf{E}}
\newcommand{\bfF}{\mathbf{F}}
\newcommand{\bfG}{\mathbf{G}}
\newcommand{\bfH}{\mathbf{H}}
\newcommand{\bfI}{\mathbf{I}}
\newcommand{\bfJ}{\mathbf{J}}
\newcommand{\K}{\mathbf{K}} % Exception (Homotopy category)
\newcommand{\bfL}{\mathbf{L}}
\newcommand{\bfM}{\mathbf{M}}
\newcommand{\Mod}{\mathbf{Mod}} % Addition (Modules)
\newcommand{\bfN}{\mathbf{N}}
\newcommand{\bfO}{\mathbf{O}}
\newcommand{\bfP}{\mathbf{P}}
\newcommand{\PSh}{\mathbf{PSh}} % Addition (Presheaves)
\newcommand{\bfQ}{\mathbf{Q}}
\newcommand{\QCoh}{\mathbf{QCoh}} % Addition (Quasi-coherent sheaves)
\newcommand{\bfR}{\mathbf{R}}
\newcommand{\bfS}{\mathbf{S}}
\newcommand{\Set}{\mathbf{Set}} % Addition (Sets)
\newcommand{\Sh}{\mathbf{Sh}} % Addition (Sheaves)
\newcommand{\bfT}{\mathbf{T}}
\newcommand{\Top}{\mathbf{Top}} % Addition (Topological spaces)
\newcommand{\bfU}{\mathbf{U}}
\newcommand{\bfV}{\mathbf{V}}
\renewcommand{\Vec}{\mathbf{Vec}} % Addition (Vector bundles)
\newcommand{\bfW}{\mathbf{W}}
\newcommand{\bfX}{\mathbf{X}}
\newcommand{\bfY}{\mathbf{Y}}
\newcommand{\bfZ}{\mathbf{Z}}

% Mathfrak for ideals
\renewcommand{\a}{\mathfrak{a}}
\renewcommand{\b}{\mathfrak{b}}
\renewcommand{\c}{\mathfrak{c}}
\renewcommand{\d}{\mathfrak{d}}
\newcommand{\e}{\mathfrak{e}}
\newcommand{\m}{\mathfrak{m}}
\newcommand{\n}{\mathfrak{n}}

%%%%%%%%%% THEOREM ENVIRONMENTS %%%%%%%%%%

% Custom colors for theorem styles below
\definecolor{darkgreen}{RGB}{0,75,0}
\definecolor{darkblue}{RGB}{0,0,75}
\definecolor{darkred}{RGB}{75,0,0}
\definecolor{linkred}{rgb}{0.7,0.2,0.2}
\definecolor{linkblue}{rgb}{0,0.2,0.6}

% Custom theorem styles (empty fields take default values)
\newtheoremstyle{darkgreentheorem}% name of the style
{}% measure of space to leave above the theorem. E.g.: 3pt
{}% measure of space to leave below the theorem. E.g.: 3pt
{\itshape}% name of font to use in the body of the theorem
{}% measure of space to indent
{\color{darkgreen}\bfseries}% name of head font
{.}% punctuation between head and body
{ }% space after theorem head; " " = normal interword space
{}% Manually specify head
\newtheoremstyle{darkbluedefinition}
{}{}{}{}{\color{darkblue}\bfseries}{.}{ }{}
\newtheoremstyle{darkredexample}
{}{}{}{}{\color{darkred}\bfseries}{.}{ }{}

% Numbered theorems
\theoremstyle{plain}
% \theoremstyle{darkgreentheorem}
\newtheorem{thm}{Theorem}
\newtheorem{lm}[thm]{Lemma}
\newtheorem{prop}[thm]{Proposition}
\newtheorem{cor}[thm]{Corollary}
\newtheorem{conj}[thm]{Conjecture}
\newtheorem{fact}[thm]{Fact}
\theoremstyle{definition}
% \theoremstyle{darkbluedefinition}
\newtheorem{defn}[thm]{Definition}
\newtheorem{defnlm}[thm]{Definition-Lemma}
% \theoremstyle{darkredexample}
\newtheorem{exa}[thm]{Example}
\theoremstyle{remark}
\newtheorem{rem}[thm]{Remark}
\newtheorem{nota}[thm]{Notation}
\newtheorem{q}[thm]{Question}
\newtheorem{exe}[thm]{Exercise}

% Custom numbered theorems
\theoremstyle{plain}
% \theoremstyle{darkgreentheorem}
\newtheorem{innercustomthm}{Theorem}
\newenvironment{cthm}[1]
    {\renewcommand\theinnercustomthm{#1}\innercustomthm}
    {\endinnercustomthm}
\newtheorem{innercustomlm}{Lemma}
\newenvironment{clm}[1]
    {\renewcommand\theinnercustomlm{#1}\innercustomlm}
    {\endinnercustomlm}
\newtheorem{innercustomprop}{Proposition}
\newenvironment{cprop}[1]
    {\renewcommand\theinnercustomprop{#1}\innercustomprop}
    {\endinnercustomprop}
\newtheorem{innercustomcor}{Corollary}
\newenvironment{ccor}[1]
    {\renewcommand\theinnercustomcor{#1}\innercustomcor}
    {\endinnercustomcor}
\newtheorem{innercustomconj}{Conjecture}
\newenvironment{cconj}[1]
    {\renewcommand\theinnercustomconj{#1}\innercustomconj}
    {\endinnercustomconj}
\newtheorem{innercustomfact}{Fact}
\newenvironment{cfact}[1]
    {\renewcommand\theinnercustomfact{#1}\innercustomfact}
    {\endinnercustomfact}
% Definitions
\theoremstyle{definition}
% \theoremstyle{darkbluedefinition}
\newtheorem{innercustomdefn}{Definition}
\newenvironment{cdefn}[1]
    {\renewcommand\theinnercustomdefn{#1}\innercustomdefn}
    {\endinnercustomdefn}
\newtheorem{innercustomdefnlm}{Definition-Lemma}
\newenvironment{cdefnlm}[1]
    {\renewcommand\theinnercustomdefnlm{#1}\innercustomdefnlm}
    {\endinnercustomdefnlm}
\newtheorem{innercustomexe}{Exercise}
\newenvironment{cexe}[1]
    {\renewcommand\theinnercustomexe{#1}\innercustomexe}
    {\endinnercustomexe}
% \theoremstyle{darkredexample}
\newtheorem{innercustomexa}{Example}
\newenvironment{cexa}[1]
    {\renewcommand\theinnercustomexa{#1}\innercustomexa}
    {\endinnercustomexa}
\theoremstyle{remark}
\newtheorem{innercustomrem}{Remark}
\newenvironment{crem}[1]
    {\renewcommand\theinnercustomrem{#1}\innercustomrem}
    {\endinnercustomrem}
\newtheorem{innercustomnota}{Notation}
\newenvironment{cnota}[1]
    {\renewcommand\theinnercustomnota{#1}\innercustomnota}
    {\endinnercustomnota}
\newtheorem{innercustomq}{Question}
\newenvironment{cq}[1]
    {\renewcommand\theinnercustomq{#1}\innercustomq}
    {\endinnercustomq}

% Unnumbered theorems
\theoremstyle{plain}
% \theoremstyle{darkgreentheorem}
\newtheorem*{uthm}{Theorem}
\newtheorem*{ulm}{Lemma}
\newtheorem*{uprop}{Proposition}
\newtheorem*{ucor}{Corollary}
\newtheorem*{uconj}{Conjecture}
\newtheorem*{ufact}{Fact}
\theoremstyle{definition}
% \theoremstyle{darkbluedefinition}
\newtheorem*{udefn}{Definition}
\newtheorem*{udefnlm}{Definition-Lemma}
% \theoremstyle{darkredexample}
\newtheorem*{uexa}{Example}
\theoremstyle{remark}
\newtheorem*{urem}{Remark}
\newtheorem*{unota}{Notation}
\newtheorem*{uq}{Question}
\newtheorem*{uexe}{Exercise}

% Cross-referencing
\crefname{thm}{theorem}{theorems}
\Crefname{thm}{Theorem}{Theorems}
\crefname{lm}{lemma}{lemmas}
\Crefname{lm}{Lemma}{Lemmas}
\crefname{prop}{proposition}{propositions}
\Crefname{prop}{Proposition}{Propositions}
\crefname{cor}{corollary}{corollaries}
\Crefname{cor}{Corollary}{Corollaries}
\crefname{conj}{conjecture}{conjectures}
\Crefname{conj}{Conjecture}{Conjectures}
\crefname{fact}{fact}{facts}
\Crefname{fact}{Fact}{Facts}
\crefname{defn}{definition}{definitions}
\Crefname{defn}{Definition}{Definitions}
\crefname{defnlm}{defition-lemma}{definition-lemmas}
\Crefname{defnlm}{Definition-Lemma}{Definition-Lemmas}
\crefname{exa}{example}{examples}
\Crefname{exa}{Example}{Examples}
\crefname{rem}{remark}{remarks}
\Crefname{rem}{Remark}{Remarks}
\crefname{nota}{notation}{notations}
\Crefname{nota}{Notation}{Notations}
\crefname{q}{question}{questions}
\Crefname{q}{Question}{Questions}
\crefname{exe}{exercise}{exercises}
\Crefname{exe}{Exercise}{Exercises}
% More cross-referencing
\crefname{cthm}{theorem}{theorems}
\Crefname{cthm}{Theorem}{Theorems}
\crefname{clm}{lemma}{lemmas}
\Crefname{clm}{Lemma}{Lemmas}
\crefname{cprop}{proposition}{propositions}
\Crefname{cprop}{Proposition}{Propositions}
\crefname{ccor}{corollary}{corollaries}
\Crefname{ccor}{Corollary}{Corollaries}
\crefname{cconj}{conjecture}{conjectures}
\Crefname{cconj}{Conjecture}{Conjectures}
\crefname{cfact}{fact}{facts}
\Crefname{cfact}{Fact}{Facts}
\crefname{cdefn}{definition}{definitions}
\Crefname{cdefn}{Definition}{Definitions}
\crefname{cdefnlm}{definition-lemma}{definition-lemmas}
\Crefname{cdefnlm}{Definition-Lemma}{Definition-Lemmas}
\crefname{cexa}{example}{examples}
\Crefname{cexa}{Example}{Examples}
\crefname{crem}{remark}{remarks}
\Crefname{crem}{Remark}{Remarks}
\crefname{cnota}{notation}{notations}
\Crefname{cnota}{Notation}{Notations}
\crefname{cq}{question}{questions}
\Crefname{cq}{Question}{Questions}
\crefname{cexe}{exercise}{exercises}
\Crefname{cexe}{Exercise}{Exercises}

%%%%%%%%%% TIKZCD %%%%%%%%%%

% Open and closed immersion arrows.
\makeatletter
\tikzcdset{
open/.code={\tikzcdset{hook, circled};},
closed/.code={\tikzcdset{hook, slashed};},
circled/.code={\tikzcdset{markwith={\draw (0,0) circle (.375ex);}};},
slashed/.code={\tikzcdset{markwith={\draw[-] (-.4ex,-.4ex) -- (.4ex,.4ex);}};},
markwith/.code={
\pgfutil@ifundefined{tikz@library@decorations.markings@loaded}%
{\pgfutil@packageerror{tikz-cd}{You need to say %
\string\usetikzlibrary{decorations.markings} to use arrow with markings}{}}{}%
\pgfkeysalso{/tikz/postaction={/tikz/decorate,
/tikz/decoration={
markings,
mark = at position 0.5 with
{#1}}}}},
}
\makeatother

%%%%%%%%%% AUTHOR, TITLE, ETC %%%%%%%%%%

% Author info
\author{Pedro Núñez}
\address{Pedro Núñez, Mathematisches Institut, Albert-Ludwigs-Universität Freiburg, Ernst-Zermelo-Straße 1, 79104 Freiburg im Breisgau, Germany}
\email{\href{mailto:pedro.nunez@math.uni-freiburg.de}{pedro.nunez@math.uni-freiburg.de}}
\urladdr{\href{https://home.mathematik.uni-freiburg.de/nunez/}{https://home.mathematik.uni-freiburg.de/nunez}}
\thanks{Supported by the DFG-Graduiertenkolleg GK1821 ``Cohomological Methods in Geometry'' at the University of Freiburg.}

% Content details
%\keywords{...}
%\subjclass[...]{...}
\title[Higgs Bundles --- Existence of Solutions]{Higgs Bundles --- Existence of Solutions}
\date{15 July 2020}

% Links and pdf options
\makeatletter
\hypersetup{
  pdfauthor={\authors},
  pdftitle={\@title},
  %pdfsubject={\@subjclass},
  %pdfkeywords={\@keywords},
  pdfstartview={Fit},
  pdfpagelayout={TwoColumnRight},
  pdfpagemode={UseOutlines},
  bookmarks,
  colorlinks,
  linkcolor=linkblue,
  citecolor=linkred,
  urlcolor=linkred}
\makeatother
 % Content described in README.md

% Math operators
\DeclareMathOperator{\Hom}{Hom}
\DeclareMathOperator{\End}{End}
\DeclareMathOperator{\Ext}{Ext}
\DeclareMathOperator{\Aut}{Aut}
\DeclareMathOperator{\Sym}{Sym}
\DeclareMathOperator{\Spec}{Spec}
\DeclareMathOperator{\Proj}{Proj}
\DeclareMathOperator{\Pic}{Pic}
\DeclareMathOperator{\rk}{rk}
\DeclareMathOperator{\tr}{tr}
\DeclareMathOperator{\pr}{pr}
\DeclareMathOperator{\ad}{ad}
\DeclareMathOperator{\GL}{GL}
\let\div\relax
\DeclareMathOperator{\div}{div}
\DeclareMathOperator{\SO}{SO}
\DeclareMathOperator{\SU}{SU}
\let\U\relax
\DeclareMathOperator{\U}{U}
\DeclareMathOperator{\SL}{SL}
\DeclareMathOperator{\su}{\mathfrak{su}}
\DeclareMathOperator{\so}{\mathfrak{so}}
\let\Im\relax
\DeclareMathOperator{\Im}{Im}
\DeclareMathOperator{\Ker}{Ker}
\DeclareMathOperator{\ch}{ch}
\DeclareMathOperator{\td}{td}
\DeclareMathOperator{\grad}{grad}

% Other commands
\newcommand{\ot}{\otimes}
\newcommand{\op}{\oplus}
\newcommand{\dual}{^{\vee}}
\newcommand{\act}{\rotatebox[origin=c]{-120}{$\circlearrowright$}}
\newcommand{\Diff}{\mathbf{Diff}}
\newcommand{\Grp}{\mathbf{Grp}}
\renewcommand{\X}{\mathfrak{X}}
\newcommand{\g}{\mathfrak{g}}
\renewcommand{\u}{\mathfrak{u}}
\newcommand{\prp}{^{\perp}}
\newcommand{\1}{\mathbbm{1}}

\begin{document}

\maketitle

\begin{abstract}
    We introduce the stability condition for Higgs bundles and discuss the Hitchin--Kobayashi correspondence \cite[\S 3 and \S 4]{hit87a}.
\end{abstract}

\tableofcontents

\begin{center}
\textcolor{gray}{---parts in gray will be omitted during the talk---}
\end{center}

\section{Introduction}

Let $M$ be a compact Riemann surface of genus $g$.
Fix a Hermitian metric on $M$ normalized to unit volume, \textcolor{gray}{i.e.~such that
\[ \operatorname{Vol}(M)=\int_{M}\w_{M}=1, \]
where $\w_{M}\in A^{1,1}(M)$ is the corresponding Kähler form \cite[Beginning of \S 3.3]{voi02} and $\operatorname{Vol}(M)$ is the Riemannian volume \cite[Lemma 3.8]{voi02}.}

Hitchin's equations \cite[(1.3)]{hit87a} were originally stated for principal $G$-bundles over $M$, but for $G=\SO(3)$ they reduce to the study of rank $2$ complex vector bundles on $M$ \cite[Beginning of \S 2]{hit87a}.
So let $E$ be a rank $2$ complex vector bundle on $M$ and fix a Hermitian metric $h$ on $E$ \cite[Proposition 4.1.4]{huy05}.
A solution to Hitchin's equations in this set up consists of a unitary connection $\nabla \colon A^{0}(E)\to A^{1}(E)$ and a traceless endomorphism valued $(1,0)$-form $\Phi\in A^{1,0}(\End_{0}(E))$ such that
\begin{equation}\label{eqn:hitchin}
    \begin{cases}
	F+[\Phi,\Phi^{*}]=-2\pi i \mu(E)\1_{E}\w_{M}, \\
	\nabla^{0,1}\Phi=0.
    \end{cases}
\end{equation}
\textcolor{gray}{%
About the notation:
\begin{itemize}
    \item $F\colon A^{0}(E)\to A^{2}(E)$ is the curvature of the connection $\nabla$.
    \item $\Phi$ can be locally written as $\phi dz$, where $\phi$ is a local section of $\End_{0}(E)$ and $z$ is a local holomorphic coordinate.
	Then $\Phi^{*}$ can be locally written as $\phi^{*} d\bar{z}$, where $\phi^{*}$ is the $h$-adjoint of $\phi$; and $[\Phi,\Phi^{*}]$ can be locally written as $[\phi,\phi^{*}]dz\wedge d\bar{z}$.
    \item $\mu(E)$ is the slope of the vector bundle $E$, defined as $\frac{1}{2}\deg(\Lambda^{2}E)$.
    \item $\1_{E}\in A^{0}(\End(E))$ denotes the identity endomorphism.
    \item $\nabla^{0,1}\colon A^{0}(E)\to A^{0,1}(E)$ denotes the $(0,1)$-part of $\nabla$.
    \item We followed the sign convention in \cite[(4.17)]{wen16}.
	Other references, such as \cite[p.~38]{tho05}, follow the opposite sign convention.
\end{itemize}
}

\textcolor{gray}{%
Since $d\bar{z}\wedge d\bar{z}=0$, the integrability condition is automatic on a Riemann surface, so $\nabla^{0,1}$ defines a holomorphic structure on $E$ \cite[Theorem 2.6.26]{huy05}.
We can regard $\Phi$ as a smooth section of $\End_{0}(E)\ot K$.
The condition $\nabla^{0,1}\Phi=0$ then translates into $\Phi$ being a holomorphic section of $\End_{0}(E)\ot K$.
This in turn can be regarded as a holomorphic morphism $E\to E\ot K$.
Moreover, if $(E,\Phi)$ is a solution which does not degenerate into a direct sum of line bundles, then every line bundle $L\subseteq $ such that $\Phi(L)\subseteq L\ot K$ verifies $\mu(L)<\mu(E)$ \cite[Proposition 4.2.16]{wen16}.
This condition is what we will call \textit{stability} of the Higgs bundle $(E,\Phi)$.
}

The existence proof in \cite[(4.3)]{hit87a} is based on Donaldson's proof of the existence of connections with constant central curvature on stable vector bundles \cite{don83}, which was discussed in Tanuj's talk.
The difference between the two settings is the introduction of the Higgs field $\Phi$, so the main goal of this talk will be to understand stability of such pairs $(E,\Phi)$.
At the end of the talk we will see how to modify Donaldson's proof to produce solutions to Hitchin's equations.

\section{Stability of Higgs bundles}

\begin{unota}
    In this first section, all vector bundles, morphisms and sections are assumed to be holomorphic.
    We will often go back and forth between vector bundles $E$ and their sheaves of sections $\E$.
    We denote the set of rank $2$ vector bundles on $M$ by $\Vec_{2}(M)$.
\end{unota}

\begin{udefn}[Higgs bundle]
    A \textit{Higgs bundle} on $M$ is a pair $(E,\Phi)$, where $E\in \Vec_{2}(M)$ and $\Phi\in H^{0}(M,\sEnd_{0}(\E)\ot \sK)$.
    We call $\Phi$ a \textit{Higgs field} on $E$.
\end{udefn}

\begin{urem}
    Using the canonical isomorphisms
    \[ H^{0}(M,\sEnd(\E)\ot \sK)\cong \Hom(\O,\E\dual \ot \E\ot \sK)\cong \Hom(\E,\E\ot \sK) \]
    we may identify $\Phi$ with a morphism $\Phi\colon E\to E\ot K$.
\end{urem}

\begin{udefn}[Stability]
    A Higgs bundle $(E,\Phi)$ is said to be \textit{stable} if for every $\Phi$-invariant line bundle $L\subseteq E$ we have $\mu(L)<\mu(E)$, where $\Phi$-invariance means that $\Phi(L)\subseteq L\ot K$.
\end{udefn}

\begin{urem}
    $(E,0)$ is stable if and only if $E$ is stable in the usual sense.
\end{urem}

\begin{cexe}{A}
    There are no stable Higgs bundles on $\P^{1}$.
    [\textit{Hints below}\footnote{Grothedieck's theorem allows us to write $\Phi$ as a matrix.
    What can we say about each entry?
    The solution can be found in \cite[(3.2) (iii)]{hit87a}}]
\end{cexe}

\begin{lm}\label{lm:traceless}
    For every $E\in \Vec_{2}(E)$ there is a short exact sequence
    \[ 0\to \O\to \sEnd(\E)\to \sEnd_{0}(\E)\to 0 \]
    in which the left hand side is the trivial line bundle generated by $\1_{E}$.
    In particular, by Hirzebruch--Riemann--Roch,
    \[ \chi(\sEnd_{0}(\E)\ot \sK)=3g-3. \]
    \begin{proof}
	Over $x\in M$ we consider the projection $\End(E_{x})\to \End_{0}(E_{x})$ is given by
	\[ \phi\mapsto \phi-\frac{\tr(\phi)}{2}\1_{E_{x}}. \]
	The endomorphisms in the kernel are precisely the multiples of the identity, and this fiber-wise description globalizes to the desired short exact sequence.	
	\textcolor{gray}{%
	For the Chern class computation we use the axioms in \cite[Appendix A]{har77}.
	We have
	\[ c_{1}(\sEnd_{0}(\E))=c_{1}(\sEnd(\E))=c_{1}(\E\dual\ot\E)=0, \]
	therefore
	\[ \ch(\sEnd_{0}(\E)\ot \sK)=3+c_{1}(\sEnd_{0}(\E))+3c_{1}(\sK)=3+3c_{1}(\sK). \]
	We also have
	\[ \td(\sK\dual)=1-\frac{c_{1}(\sK)}{2}, \]
	so multiplying the two expressions we obtain
	\[ \ch(\sEnd_{0}(\E)\ot \sK)\td(\sK\dual)=3+\frac{3}{2}c_{1}(\sK). \]
	Since $\deg(c_{1}(\sK))=2g-2$, we deduce that
	\[ \deg(\ch(\sEnd_{0}(\E)\ot\sK)\td(\sK\dual))_{1}=3g-3. \]
	The second assertion follows now from Hirzebruch--Riemann--Roch \cite[Theorem A.4.1]{har77}.
	}
    \end{proof}
\end{lm}

\begin{unota}
    We define the following subsets of $\Vec_{2}(M)$:
    \begin{itemize}
	\item $\bfS:=\{ E\in \Vec_{2}(M)\mid \exists \Phi \text{ such that }(E,\Phi)\text{ is stable}\}$.
	\item $\bfA:=\{ E\in \Vec_{2}(M)\mid \text{for general }\Phi\text{ there is no invariant }L \}$.
	\item $\bfB:=\{ E\in \Vec_{2}(M)\mid \exists L\text{ invariant for all }\Phi\}$.
    \end{itemize}
\end{unota}

\begin{lm}\label{lm:twocases}
    If $g\geqslant 2$, then
    \[ \Vec_{2}(M)=\bfA\sqcup \bfB. \]
    \begin{proof}
	Let $\pi\colon \P(E)\to M$ be the projectivization of our rank $2$ vector bundle and let $O(-1)\to \P(E)$ denote the tautological line bundle, whose fiber over $[v]\in \P(E)$ is the line $\{\lambda v\mid \lambda\in \C\}\subseteq E_{\pi([v])}$ spanned by $v$.
	Let $O(l):=O(-1)^{\ot (-l)}$.
	If $\F$ is a sheaf on $\P(E)$, we denote $\F(l):=\F\ot \O(l)$, where $\O(l)$ denotes the sheaf of sections of $O(l)$.
	Using the relative $\Proj$ construction we can write $\P(E)=\P(\E\dual):=\Proj_{M}(S(\E\dual))$, so \cite[Exercise III.8.4]{har77} implies that
	\[ \pi_{*}\O(l)=\begin{cases} S^{l}(\E\dual) & \text{ if } l\geqslant 0, \\
	    0 & \text{ if }l<0.
	\end{cases}
	\]

	Let $x\in M$.
	Given $\phi\in \End(E_{x})$, we define the quadratic form $v\mapsto \phi(v)\wedge v$ with values in $\Lambda^{2}E_{x}$, which corresponds to an element in $S^{2}(E_{x}\dual)\ot\Lambda^{2}E_{x}$.
	The resulting quadratic form is trivial precisely when $\phi=\lambda\1_{E_{x}}$ for some $\lambda\in \C$, so by \Cref{lm:traceless} we obtain an injective homomorphism $\End_{0}(E_{x})\to S^{2}(E_{x}\dual)\ot \Lambda^{2}E_{x}$.
	Both vector spaces have the same dimension, so this must be an isomorphism.
	These isomorphisms globalize to an isomorphism $\End_{0}(E)\cong S^{2}(E\dual)\ot \Lambda^{2}E$, hence we obtain an isomorphism
	\[ \sEnd_{0}(\E)\ot \sK\cong \pi_{*}\O(2)\ot (\Lambda^{2}\E)\ot \sK. \]
	The projection formula yields now an isomorphism $\sEnd_{0}(\E)\ot \sK\cong \pi_{*}(\pi^{*}(\sK\ot \Lambda^{2}\E)(2))$, hence an isomorphism
	\[ \psi \colon H^{0}(M,\sEnd_{0}(\E)\ot \sK)\cong H^{0}(\P(E),\pi^{*}(\sK\ot \Lambda^{2}\E)(2)). \]
	Let now $\Phi\in H^{0}(M,\sEnd_{0}(\E)\ot \sK)$.
	By construction of $\psi$, a line bundle $L\subseteq E$ is $\Phi$-invariant if and only if $\psi(\Phi)$ vanishes at all $[v]\in \P(L)\subseteq \P(E)$.
	In other words, $L$ is $\Phi$-invariant if and only if $\sigma(M)\subseteq \div(\psi(\Phi))$, where $\div(-)$ denotes the divisor of zeros of a section and $\sigma\colon M=\P(L)\to \P(E)$ is the section induced by $L\subseteq E$.

	Suppose now that $\Phi\in H^{0}(M,\sEnd_{0}(\E)\ot \sK)$ is not nilpotent and let $L\subseteq E$ be a $\Phi$-invariant line bundle.
	Over a general point $x\in M$ the corresponding traceless endomorphism $\phi\in \End_{0}(E_{x})$ is diagonalizable, so we can find some eigenvector $v\in E_{x}\setminus L_{x}$ in an eigenspace other than $L_{x}$.
	This gives us a point $[v]\in \P(E)\setminus \sigma(M)$ on which $\psi(\Phi)$ vanishes.
	Hence $\sigma(M)$ is a proper irreducible component of the divisor $\div(\psi(\Phi))$.

	The previous discussion shows that if $\Phi$ is not nilpotent and $\div(\psi(\Phi))$ is irreducible, then there are no $\Phi$-invariant line bundles $L\subseteq E$.
	By \Cref{lm:traceless} we have
	\[ h^{0}(M,\sEnd_{0}(\E)\ot \sK)\geqslant 3g-3\geqslant 3, \]
	so the complete linear system defined by the invertible sheaf $\pi^{*}(\sK\ot\Lambda^{2}\E)(2)$ has dimension at least $2$.
	If this linear system does not have a fixed divisor, then Bertini's theorem \cite[Theorem 7.19]{iit82} tells us that $\div(\psi(\Phi))$ is irreducible for a general $\Phi\in H^{0}(M,\sEnd_{0}(\E)\ot \sK)$.
	Since in our case $\Phi\in H^{0}(M,\sEnd_{0}(\E)\ot \sK)$ is nilpotent if and only if $\Phi^{2}=0$, a general $\Phi$ is not nilpotent.
	Therefore $E\in \bfA$ in this case.

	Let us see what happens if the linear system has a fixed divisor.
	Such a fixed divisor must be the zero locus of a non-zero global section $s$ of an invertible sheaf on $\P(E)$, which is up to isomorphism of the form $\pi^{*}\L(l)$ with $\L$ an invertible sheaf on $M$ and $l\in \Z$ \cite[Exercise II.7.9]{har77}.
	Being a fixed divisor means then that every other global section of our line bundle can be written as a product $st$, where $t\in H^{0}(\P(E),\pi^{*}\sN(2-l))$.
	Since our line bundle had non-zero global sections, both $\pi^{*}\L(l)$ and $\pi^{*}\sN(2-l)$ must have non-zero global sections.
	By the projection formula, this leaves us with only three possibilities:
	\begin{enumerate}[label=\alph*)]
	    \item $l=0$;
	    \item $l=1$;
	    \item $l=2$.
	\end{enumerate}
	We analyze each case separately to conclude that
	\[ E\in \begin{cases} \bfA & \text{ if }l=0, \\
	    \bfB & \text{ if }l\in \{1,2\}.
	\end{cases} \]

	\color{gray}
	Let us start with case $a)$.
	Let $\pi^{*}s\in H^{0}(\P(E),\pi^{*}\L)\cong H^{0}(M,\L)$ be a global section corresponding to the fixed component of our linear system.
	Dividing all global sections by $s$ and by $\pi^{*}s$ respectively we obtain the following commutative diagram:
	\begin{center}
	    \begin{tikzcd}
		H^{0}(M,\sEnd_{0}(\E)\ot \sK)\arrow{r}{\cong}\arrow[swap]{d}{/s}[swap]{\cong} & H^{0}(\P(E),\pi^{*}(\sK\ot \Lambda^{2}\E)(2))\arrow{d}{\cong}[swap]{/\pi^{*}s} \\
		H^{0}(M,\sEnd_{0}(\E)\ot \sK\ot \L\dual)\arrow{r}{\cong} & H^{0}(\P(E),\pi^{*}(\L\dual \ot \sK\ot \Lambda^{2}\E)(2))
	    \end{tikzcd}
	\end{center}
	By definition, the new linear system does not have any fixed divisors and has the same dimension.
	Hence we can apply Bertini to conclude that a generic $\Phi'\in H^{0}(M,\sEnd_{0}(\E)\ot \sK\ot \L\dual)$ does not have invariant line bundles, which in this case are defined as line bundles $N\subseteq E$ such that $\Phi'(N)\subseteq N\ot K\ot L\dual$.
	But a line bundle $N\subseteq E$ is $\Phi'$-invariant if and only if it is $s\Phi'$-invariant, so we have $E\in \bfA$ in this case.

	We move on to case $b)$.
	Assume that the fixed divisor corresponds to a non-zero global section $s\in H^{0}(\P(E),\pi^{*}\L(1))$.
	This corresponds to a non-zero morphism $E\to L$.
	The fiber-wise kernel has then dimension $1$ generically and $2$ at special points by upper semi-continuity \cite[Proposition 1.3.2]{ati89}.
	Let $N\subseteq E$ be the largest line subbundle of $E$ contained in the kernel of $s$.
	If $v\in N$ is a non-zero vector, then $s(v)=0$ and so $[v]\in \div(s)\subseteq \div(\psi(\Phi))$.
	Thus the corresponding section $\sigma(M)\subseteq \P(E)$ is contained in $\div(\psi(\Phi))$ for all $\Phi$ and $N$ is $\Phi$-invariant for all $\Phi$.
	Hence $E\in \bfB$ in this case.

	In case $c)$, the fixed divisor corresponds to a non-zero global section of $\pi^{*}\L(2)$.
	We have
	\[ H^{0}(\P(E),\pi^{*}\L(2))\cong H^{0}(M,\L\ot S^{2}\E\dual)\cong H^{0}(M,\sEnd_{0}(\E)\ot \L\ot \Lambda^{2}\E\dual). \]
	So we can think of the fixed global section $s$ as a traceless endomorphism of $E$ with coefficients in $L\ot \Lambda^{2}E\dual$.
	With this point of view, $s$-invariance of a line bundle $N\subseteq E$ translates into $s\Phi'$-invariance of $N\subseteq E$ as before, where $s\Phi'$ is a Higgs field.
	Let us see that the fixed section $s$ has some non-trivial kernel, hence defining a line bundle invariant under all Higgs fields as in the previous case.
	To show that there is some non-trivial kernel, it suffices to check that $\det(s)=0$.
	Since $s$ is traceless, it suffices in turn to check that $\tr(s^{2})=0$.
	Suppose on the contrary that $\tr(s^{2})\neq 0$.
	Fix some non-zero $s_{1}\in H^{0}(M,\L\dual\ot \sK\ot \Lambda^{2}\E)$ and consider the linear map
	\begin{align*}
	    \theta \colon H^{0}(M,\L\dual\ot\sK\ot \Lambda^{2}\E) & \longrightarrow H^{0}(M,\sK^{2}) \\
	    \Phi' & \longmapsto \tr(s^{2})s_{1}\Phi'
	\end{align*}
	Since $\tr(s^{2})s_{1}$ can only vanish at finitely many points, the image of a non-zero $\Phi'$ can only vanish at finitely many points, hence $\theta$ is injective.
	From Hirzebruch--Riemann--Roch and \Cref{lm:traceless} we know that
	\[ h^{0}(M,\L\dual\ot\sK\ot \Lambda^{2}\E)\geqslant 3g-3=h^{0}(M,\sK^{2}), \]
	so $\theta$ is an isomorphism.
	Since $\L\dual\ot \sK\ot \Lambda^{2}\E$ has global sections, its degree is non-negative.
	If it was zero, then this would be the trivial line bundle and we would have $h^{0}(M,\L\dual\ot\sK\ot\Lambda^{2}\E)=1<3g-3$, a contradiction.
	Hence $\deg(\L\dual\ot\sK\ot\Lambda^{2}\E)>0$ and the non-zero global section $s_{1}$ has at least one zero.
	If $\theta$ was indeed an isomorphism, then each zero of $s_{1}$ would give a base point of the complete linear system corresponding to $\sK^{2}$.
	But $\deg(\sK^{2})=4g-4\geqslant 2g$, so this linear system has no base points \cite[Corollary IV.3.2]{har77}.
	This contradiction shows that $s$ has non-trivial kernel, which contains a line bundle $N\subseteq E$ invariant by all $\Phi\in H^{0}(M,\sEnd_{0}(\E)\ot \sK)$.
	Hence $E\in \bfB$ as well in this case.
    \end{proof}
\end{lm}

\begin{cexe}{B}
    Assume $g\geqslant 2$.
    Let $K^{\frac{1}{2}}$ be a line bundle whose square is $K$ and let $K^{-\frac{1}{2}}$ be its inverse.
    Does $K^{\frac{1}{2}}\op K^{-\frac{1}{2}}$ belong to $\bfA$ or to $\bfB$?
    [\textit{Hints below}\footnote{Consider the family of traceless endomorphisms given by 
    $$\Phi_{\alpha}:=\begin{pmatrix} 0 & \alpha \\ 1 & 0 \end{pmatrix},$$
	parametrized by quadratic differentials $\alpha\in H^{0}(M,K^{2})$. Use without proof the fact that an invariant line bundle exists if and only if the characteristic polynomial
	$$\lambda^{2}-\alpha$$
	has a root in $H^{0}(M,K)$, i.e.~if and only if the quadratic differential $\alpha$ can be written as a square $\alpha=\beta^{2}$ for some differential form $\beta\in H^{0}(M,K)$. If $\alpha$ was a square, its zeros would all have multiplicity at least two. Conclude that $K^{\frac{1}{2}}\op K^{-\frac{1}{2}}\in \bfA$ using Bertini's theorem.}]
\end{cexe}

\begin{lm}\label{lm:linearalgebra}
    Let $E\in \Vec_{2}(M)$ and $L\subseteq E$ a line bundle.
    We have the following short exact sequences:
    \begin{enumerate}[label=\alph*)]
	\item $0\to \L^{2}\ot (\Lambda^{2}\E)\dual \ot \sK \to \sEnd_{0}(\E)\ot \sK\to \E\ot \L^{-1}\ot \sK\to 0$.
	\item $0\to \E\dual \ot \L\ot \sK\to \sEnd_{0}(\E)\ot \sK \to \L^{-2}\ot (\Lambda^{2}\E)\ot \sK\to 0$.
    \end{enumerate}
    Moreover, the non-zero sections of the image of $\L\ot (\Lambda^{2}\E\dual)\ot \sK$ in $a)$ leave only $L$ invariant; and the sections of the image of $\E\dual\ot \L\ot \sK$ in $b)$ are those which leave at least $L$ invariant.
    \color{gray}
    \begin{proof}
	All the short exact sequences are the result of tensoring another short exact sequence with $\sK$, so let us find the necessary short exact sequences without $\sK$.
	Under the isomorphism $E\cong E\dual \ot \Lambda^{2}E$ \cite[Exercise II.5.16]{har77}, the line bundle $L$ is sent to linear forms with coefficients in $\Lambda^{2}E$ vanishing along $L$, hence we have a short exact sequence
	\[ 0\to \L\to \E\to \L\dual \ot \Lambda^{2}\E\to 0. \]
	Dualizing this short exact sequence we obtain an inclusion $L\ot \Lambda^{2}E\dual \subseteq E\dual$.
	Tensoring with $L$ and composing with the inclusion $E\dual\ot L\subseteq E\dual \ot E$ we obtain an inclusion $L^{2}\ot \Lambda^{2}E\dual \subseteq \End(E)$.
	Choosing a basis on each fiber and chasing all the identifications we have made so far, we see that the image of $L^{2}\ot \Lambda^{2}E\dual$ lies actually in $\End_{0}(E)$.
	Indeed, let $V$ be a two dimensional $\C$-vector space and let $e_{1}$ and $e_{2}$ be a basis.
	Let $L$ be the line spanned by a non-zero vector $l$, which we may assume to be $e_{1}$.
	The first identification we have is $V\cong \Hom(V,\Lambda^{2}V)$, sending $v$ to the homomorphism $v'\mapsto v'\wedge v$.
	This corresponds to $\alpha_{v}\ot (e_{1}\wedge e_{2})\in V\dual\ot\Lambda^{2}V$, where $\alpha_{v}\in V\dual$ is the linear form sending $e_{1}\mapsto v_{2}$ and $e_{2}\mapsto -v_{1}$.
	Denoting by $\overline{\alpha_{v}}$ its image in $L\dual$, we can describe the morphism corresponding to the right hand side of the previous short exact sequence as
	\begin{align*}
	    V &\longrightarrow L\dual \ot \Lambda^{2}V \\
	    v &\longmapsto \overline{\alpha_{v}}\ot (e_{1}\wedge e_{2})
	\end{align*}
	Let now $\beta\in (L\dual \ot \Lambda^{2}V)\dual$ and denote by $\lambda_{v}^{\beta}$ the complex number such that
	\[ \overline{\alpha_{v}}\ot (e_{1}\wedge e_{2})\overset{\beta}{\longmapsto} \lambda_{v}^{\beta}. \]
	A point $\mu l\ot \beta\in L\ot (L\dual\ot \Lambda^{2}V)\dual$ corresponds then to the endomorphism
	\begin{align*}
	    V &\longrightarrow V \\
	    v &\longmapsto \mu \lambda_{v}^{\beta}l
	\end{align*}
	A basis for $L$ is $e_{1}$, a basis for $L\dual \ot \Lambda^{2}V$ is $\overline{\alpha_{e_{2}}}\ot (e_{1}\wedge e_{2})$ and a basis for $L\ot (L\dual\ot \Lambda^{2})\dual$ is $e_{1}\ot \beta_{0}$, where $\beta_{0}\in (L\dual \ot \Lambda^{2}V)\dual$ is such that $\lambda_{e_{2}}^{\beta_{0}}=1$.
	Writing the image of the basis $e_{1}\ot \beta_{0}$ under the map $L\ot (L\dual\ot \Lambda^{2}V)\dual\to \End(V)$ as a matrix with respect to our bases we obtain
	\[
	    \begin{pmatrix}
		0 & 1 \\
		0 & 0
	    \end{pmatrix},
	\]
	because $\overline{\alpha_{e_{1}}}=0$ and therefore $\lambda_{e_{1}}^{\beta}=0$ for any $\beta$.
	We have thus the desired injective homomorphism
	\[ L^{2}\ot \Lambda^{2}V\dual\hookrightarrow \End_{0}(V) \]
	whose image are the traceless endomorphisms which preserve only $L$.
	This is the morphism from which we obtain the short exact sequence in $a)$.

	We regard this as a homomorphism into $\End(V)$ for a moment and use the basis $e_{11},e_{12},e_{21},e_{22}$ of $\End(V)$, where $e_{ij}$ denotes the endomorphism which, represented as a matrix in terms of our basis, has zeros everywhere except for a $1$ in the $ij$-th position.
	Then our homomorphism is given by
	\[ 
	    \begin{pmatrix}
		0 \\
		1 \\
		0 \\
		0
	    \end{pmatrix}
	\]
	Dualizing it we obtain a surjection
	\[ \End_{0}(V\dual)\twoheadrightarrow L^{-2}\ot \Lambda^{2}V \]
	given with respect to the dual bases by
	\[
	    \begin{pmatrix}
		0 & 1 & 0 & 0
	    \end{pmatrix}.
	\]
	Its kernel are the endomorphisms of $V\dual$ represented with respect to the dual basis by a matrix of the form
	\[
	    \begin{pmatrix}
		a & 0 \\
		b & -a
	    \end{pmatrix}.
	\]
	Hence, under the isomorphism $\End(V)\cong \End(V\dual)$ given in coordinates by sending a matrix to its transpose, we obtain a surjection
	\[ \End_{0}(V)\twoheadrightarrow L^{-2}\ot \Lambda^{2}V \]
	whose kernel are endomorphisms represented with respect to our basis by matrices of the form
	\[
	    \begin{pmatrix}
		a & b \\
		0 & -a
	    \end{pmatrix}.
	\]
	Therefore the kernel of this surjection consists precisely of the traceless endomorphisms of $V$ that leave at least $L$ invariant.
	The inclusion of this kernel can be naturally regarded as the composition of the inclusion $\Hom(V,L)\subseteq \Hom(V,V)$ and the projection $\pr_{0}\colon \Hom(V,V)\to \End_{0}(V)$, which writing every homomorphism as a matrix with respect to the bases above has the form
	\[ 
	\begin{pmatrix}
	    a & b
	\end{pmatrix}
	\mapsto
	\begin{pmatrix}
	    a & b \\
	    0 & 0
	\end{pmatrix}
	\mapsto
	\begin{pmatrix}
	    \frac{a}{2} & b \\
	    0 & -\frac{a}{2}
	\end{pmatrix}.
	\]
	This gives us the short exact sequence in $b)$ and the right hand side of the short exact sequence in $a)$.
    \end{proof}
\end{lm}

\begin{lm}\label{lm:stable}
    If $g\geqslant 2$ and $E$ is a stable rank $2$ vector bundle, then $E\in \bfA$.
    \begin{proof}
	By \Cref{lm:twocases} it suffices to show that it is not in $\bfB$.
	So let $E$ be a stable rank $2$ vector bundle on $M$ and assume $L\subseteq E$ is a line bundle which is $\Phi$-invariant for all $\Phi\in H^{0}(M,\sEnd_{0}(\E)\ot \sK)$.
	Consider the short exact sequence $b)$ from \Cref{lm:linearalgebra}
	\[ 0\to \E\dual\ot \L\ot\sK\to \sEnd_{0}(\E)\ot \sK \to \L^{-2}\ot (\Lambda^{2}\E)\ot\sK\to 0. \]
	Since all $\Phi\in H^{0}(M,\sEnd_{0}(\E)\ot\sK)$ leave $L$ invariant, we get an induced isomorphism on global sections $H^{0}(M,\E\dual\ot\L\ot\sK)\cong H^{0}(M,\sEnd_{0}(\E)\ot \sK)$.
	The left hand side of the short exact sequence $a)$ in \Cref{lm:linearalgebra} factors by construction into a short exact sequence of the form 
	\[ 0\to \L^{2}\ot (\Lambda^{2}\E)\dual\ot\sK \to \E\dual \ot \L\ot \sK\to \sQ\ot\sK\to 0, \]
	where $\deg(\sQ)$ can be seen to be $0$ by computing first Chern classes.
	Riemann--Roch says then that $h^{0}(\sQ\ot\sK)\in \{g-1,g\}$.
	Since $E$ is stable, we have $\deg(\L^{2}\ot \Lambda^{2}\E\dual)<0$, and since the complete linear system corresponding to $\sK$ is base-point free \cite[Lemma IV.5.1]{har77}
	    we have $h^{0}(M,\L^{2}\ot(\Lambda^{2}\E)\dual\ot\sK)\leqslant g-1$ by \cite[Proposition IV.3.1]{har77}.
	The long exact sequence of the previous short exact sequence gives then
	\[ h^{0}(M,\E\dual\ot\L\ot\sK)\leqslant 2g-1. \]
	The earlier Hirzebruch--Riemann--Roch computation showed that
	\[ h^{0}(M,\sEnd_{0}(\E)\ot \sK)\geqslant 3g-3. \]
	If we want the two dimensions to be equal we must have $g=2$ and $h^{0}(M,\sK\ot\L\ot\E\dual)=3$.
	From the same long exact sequence as before we deduce, using that $h^{0}(M,\L^{2}\ot (\Lambda^{2}\E)\dual\ot \sK)<g=2$, that $h^{0}(M,\L^{2}\ot(\Lambda^{2}\E)\dual\ot\sK)=1$.
	In particular, $\deg(\L^{2}\ot(\Lambda^{2}\E)\dual\ot\sK)\geqslant 0$.
	We have $\deg(\sK)=2$ and by stability we had $\deg(\L^{2}\ot \Lambda^{2}\E\dual)<0$, so we must have
	\[ \deg(\L^{2}\ot (\Lambda^{2}\E)\dual\ot\sK)\in \{0,1\}. \]
	We treat each case separately, showing that both of them lead to contradiction.

	\color{gray}
	If it is $0$, then the existence of global sections implieis that it is the trivial line bundle, hence the previous short exact sequence becomes
	\[ 0\to \O\to \E\dual\ot\L\ot\sK\to \sK\to 0. \]
	Split short exact sequences are preserved by dualizing and tensoring with line bundles, so if this sequence was split then $\E$ would be decomposable as a direct sum of line bundles.
	But this would give more endomorphisms of $\E$ than there should be, since stable vector bundles are simple.
	So the previous short exact sequence is a non-trivial extension, hence its extension class $\xi$, defined as the image of $\1_{\sK}$ under the coboundary map $\delta\colon \Hom(\sK,\sK)\to \Ext^{1}(\sK,\O)$, is non-zero \cite[Exercise III.6.1]{har77}.
	The coboundary map $H^{0}(M,\sK)\to H^{1}(M,\O)$ is given by cup product with $\xi\in H^{1}(M,\sT_{M})$ \cite[Theorem II.7.1]{bre97}.
	Hence it is non-zero, and the long exact sequence implies then that $h^{0}(M,\E\dual\ot \sK\ot\L)\leqslant 2$, contradicting our previous conclusion that this dimension was $3$.

	If it is $1$, again by the existence of a non-zero global secction we deduce that it is the line bundle corresponding to some point $x\in M$.
	The short exact sequence becomes
	\[ 0\to \O(x)\to \E\dual\ot\L\ot\sK\to \sK\to 0, \]
	and the extension class $\xi\in H^{1}(M,\sK^{-1}(x))$ is again non-zero.
	Let $s\in H^{0}(M,\sK(-x))$ be a non-zero global section, which exists because subtracting a point makes the dimension of the space of global sections go down at most by one, and in this case precisely by one by base point freeness.
	Since $h^{0}(M,\sK)=h^{0}(M,\sK^{2}(-x))$, we can write every section in $H^{0}(M,\sK^{2}(-x))$ as $st$ for some $t\in H^{0}(M,\sK)$.
	By Serre duality, the cup product gives us a perfect pairing
	\[ H^{0}(M,\sK^{2}(-x))\times H^{1}(M,\sK^{-1}(x))\to H^{1}(M,\sK)\cong \C. \]
	Since $\xi\neq 0$, we can find some $st\in H^{0}(M,\sK^{2}(-x))$ such that $st\cup \xi\neq 0$.
	But the cup product is associative \cite[Proposition II.7.3]{bre97} and the product of section $st$ is by definition their cup product, so we must have $t\cup \xi\neq 0$.
	Therefore the coboundary map $\delta\colon H^{0}(M,\sK)\to H^{1}(M,\O)$ is non-zero, because it is given by cup product with $\xi$ \cite[Theorem II.7.1]{bre97} and so $\delta(t)=t\cup \xi \neq 0$.
	This contradicts again our previous conclusion that $h^{0}(M,\E\dual\ot \L\ot \sK)=3$.
    \end{proof}
\end{lm}

\begin{prop}\label{prop:general}
    If $g\geqslant 2$, then
    \[ \bfS=\bfA. \]
    \begin{proof}
	By definition $\bfA\subseteq \bfS$, so let us see the other inclusion.
	Let $E\in \bfS$.
	If $E$ is stable, then $E\in \bfA$ by \Cref{lm:stable}.
	So assume there exists $L\subseteq E$ such that $\mu(L)\geqslant \mu(E)$.
	Then $\deg(\L^{2}\ot(\Lambda^{2}\E)\dual\ot\sK)\geqslant 2g-2$, so Riemann--Roch implies that $h^{0}(\L^{2}\ot(\Lambda^{2}\E)\dual\ot\sK)\in \{g-1,g\}$.
	In particular, from the short exact sequence $a)$ in \Cref{lm:linearalgebra} we deduce that there exists a non-zero Higgs field leaving only $L$ invariant.
	So if $E\in \bfB$, then this $L$ is a line bundle invariant by all $\Phi\in H^{0}(M,\sEnd_{0}(\E)\ot \sK)$.
	Since $\mu(L)\geqslant \mu(E)$, this contradicts the assumption that $E\in \bfS$.
    \end{proof}
\end{prop}

\textcolor{gray}{%
Summarizing what we have seen in this first section, if $M$ is a compact Riemann surface of genus $g\geqslant 2$ and $E$ is a rank $2$ vector bundle on $M$, then there are only two mutually exclusive possibilities:
    \begin{itemize}
	\item A general section $\Phi\in H^{0}(M,\sEnd_{0}(\E)\ot \sK)$ leaves no line bundle $L\subseteq E$ invariant.
	\item There exists a line bundle $L\subseteq E$ which is invariant under all sections $\Phi\in H^{0}(M,\sEnd_{0}(\E)\ot \sK)$.
    \end{itemize}
    Moreover, $E$ appears in a stable pair $(E,\Phi)$ if and only if a general section $\Phi\in H^{0}(M,\sEnd_{0}(\E)\ot\sK)$ leaves no line bundle $L\subseteq E$ invariant.
}

\section{Idea of the existence theorem}

\begin{unota}
    From now on, all vector bundles, morphisms and sections are only assumed to be smooth unless otherwise stated.
    As in the introduction, we follow the notation in \cite{kob87} for differential forms, e.g.~$A^{p,q}(E)$ denotes the space of $(p,q)$-forms over $M$ with values in $E$.
    If $h$ is a Hermitian metric on $E$ and $\bar{\partial}_{E}\colon A^{0}(E)\to A^{0,1}(E)$ is holomorphic structure on $E$, we will denote by $\nabla(h;\bar{\partial}_{E})$ the corresopnding Chern connection \cite[Proposition 4.2.14]{huy05}.
\end{unota}

Assume from now on that $g\geqslant 2$.
Let $E\to M$ be a complex vector bundle of rank $2$ and let $h$ be a Hermitian metric on $E$.
Let $\sA_{E}$ denote the space of unitary connections on $E$.
Our goal is to find a solution $(\nabla,\Phi)\in N:=\sA_{E}\times A^{1,0}(\End_{0}(E))$ to \Cref{eqn:hitchin} starting from a stable Higgs bundle $(\nabla_{0},\Phi_{0})$, meaning that $\nabla^{0,1}_{0}$ endows $E$ with a holomorphic structure such that $(E,\Phi_{0})$ is stable.

We can endow $N$ with a Kähler structure given by
\[ \w_{N}((A_{1},\Phi_{1}),(A_{2},\Phi_{2}))=-\int_{M}\tr(A_{1}\wedge A_{2})+2i\Im(\tr(\Phi_{1}\Phi_{2}^{*})) \]
for $A_{1},A_{2}\in A^{1}(\u_{h}(E))$ and $\Phi_{1},\Phi_{2}\in A^{1,0}(\End_{0}(E))$, where $\u_{h}(E)$ is the vector bundle of skew-hermitian endomorphisms of $E$ \cite[Corollary 4.2.11]{huy05}.

The gauge group $\G=A^{0}(\U_{h}(E))$ of unitary automorphisms of $E$ acts on $N$ by conjugation on each factor, and this action admits a moment map
\[ \mu(\nabla,\Phi)=-F-[\Phi,\Phi^{*}]-2\pi i \mu(E)\1_{E}\w_{M}. \]
Strictly speaking, we have given an element in $A^{2}(\u_{h}(E))$ rather than in the dual of the Lie algebra $A^{0}(\u_{h}(E))\dual$, but we identify these two spaces via the integration pairing.
If $\xi\in A^{0}(\u_{h}(E))$ is a skew-hermitian endomorphism of $E$, then we denote by $\mu_{\xi}\in A^{0}(N)$ the smooth function obtained from composing the moment map with evaluation at $\xi$ via this integration pairing.

Recall from Vincent's talk that being a moment map means that we have an equality
\[ d\mu_{\xi}=\w_{N}(\xi_{N},-) \]
in $A^{1}(N)$, where $\mu_{\xi}\colon N\to \R$ is the composition of $\mu\colon N\to A^{2}(\u_{h}(E))=A^{0}(\u_{h}(E))\dual$ with the evaluation at $\xi$ and $\xi_{N}\in \X(N)$ is the vector field corresponding to $\xi$ under the infinitesimal action.
But with the notation from \cite[\S 3]{voi02}, the Kähler condition tells us that
\[ d\mu_{\xi}=\w_{N}(\xi_{N},-)=g(I\xi_{N},-), \]
so $\grad(\mu_{\xi})=I\xi_{N}$ by definition \cite[p.~343]{lee13}.
The gauge action extends to the complex gauge group $\G^{c}$ by the formula
\[ \psi\cdot (\nabla,\Phi)=(\nabla(h;\psi\nabla^{0,1}\psi^{-1}),\psi\Phi \psi^{-1}). \]
From now on we restrict our attention to the action of the complex gauge group $\G^{c}_{0}$ of automorphisms of $E$ with determinant $1$.
By \cite[(3.15)]{hit87a} from Jin's talk, the action of $\G^{c}_{0}$ is free on the orbit of the stable pair $(\nabla_{0},\Phi_{0})$.
Therefore the vector field $I\xi_{N}=(i \xi)_{N}$ obtained from the complex infinitesimal action is non-zero for all $\xi\neq 0$ and all $x$ in the orbit.

Define a smooth function $f\colon N\to \R_{\geqslant 0}$ by
\[ f(\nabla,\Phi):=\int_{M}||\mu(\nabla,\Phi)||^{2} \]
and restrict it to the $\G^{c}_{0}$-orbit of $(\nabla_{0},\Phi_{0})$ we have a smooth function $f_{0}\colon \G^{c}_{0}\to \R_{\geqslant 0}$ given by $f_{0}(\psi)=f(\psi\cdot (\nabla_{0},\Phi_{0}))$.
Suppose that we manage to find a minimum $\psi\cdot (\nabla_{0},\Phi_{0})$ of $f_{0}$, so that the differential $df_{0}$ vanishes at $\psi$ \cite[Exercise 11.24]{lee13}.
But since
\[ d(\langle \mu,\mu\rangle)=\langle d\mu,\mu\rangle+\langle \mu,d\mu\rangle =2\langle \mu, d\mu\rangle \]
and $\grad(\mu)(\psi\cdot (\nabla_{0},\Phi_{0}))\neq 0$, we must have $\mu(\psi\cdot (\nabla_{0},\Phi_{0}))=0$.
Therefore $\psi \cdot (\nabla_{0},\Phi_{0})$ would be a solution to \Cref{eqn:hitchin} in the $\G^{c}_{0}$-orbit of our original stable pair $(\nabla_{0},\Phi_{0})$.

Let then $(\nabla_{0},\Phi_{0})\in N$ be a stable Higgs bundle, where $\nabla_{0}=\nabla(\bar{\partial}_{E})$ is the Chern connection of the given holomorphic structure $\bar{\partial}_{E}$ on $E$.
The complex gauge group $\G^{c}_{0}$ of automorphisms of $E$ with determinant $1$ acts on $N$ as well, sending $(\nabla,\Phi)$ to $(\nabla(g\nabla^{0,1}g^{-1}),g\Phi g^{-1})$.
In particular, $\Phi$ remains a holomorphic section with respect to the new $\bar{\partial}_{E}=\nabla^{0,1}$.
Moreover, $\G^{c}_{0}$ acts freely on the orbit of the stable Higgs bundle $(\nabla_{0},\Phi_{0})$ by proposition \cite[(3.15)]{hit87a} from Jin's talk.
It follows from this and the discussion in \cite[\S 4]{hit87a} or in \cite[\S 8]{ab83} that if we manage to find a minimum of for
in the orbit of $(\nabla_{0},\Phi_{0})$, we will have solved \Cref{eqn:hitchin}.

To do this, we pick a minimizing sequence $(\nabla_{n},\Phi_{n})$ for $f$ in the orbit of $(\nabla_{0},\Phi_{0})$.
This is always possible by definition of infimum and convexity of $f$ \cite[\S 8]{ab83}.
After some analysis we can apply Uhlenbeck's theorem, which allows us to assume that $(\nabla_{n},\Phi_{n})$ has a weakly convergent subsequent.
We have to show that its limit $(\nabla,\Phi)$ is still in the same $\G^{c}_{0}$-orbit of $(\nabla_{0},\Phi_{0})$.
Let $\psi_{n}\in \G^{c}_{0}$ map $(\nabla_{n},\Phi_{n})$ to $(\nabla_{0},\Phi_{0})$.
Doing some more analysis, we may assume that the sequence $\psi_{n}$ is weakly convergent, having then a non-zero limit $\psi$ \cite[p.~82]{hit87a}.
By definition we have identities
\[ \nabla_{(n,0)}^{0,1}\psi_{n}=0 \text{ and } \Phi_{0}\psi_{n}=\psi_{n}\Phi_{n}, \]
where $\nabla_{(n,0)}$ denotes the connection induced by $\nabla_{n}$ and $\nabla_{1}$ on $\End(E)$ \cite[Example 4.2.6 iii)]{huy05}.
These remain true in the limit, so $\psi$ is a holomorphic morphism from $(E,\nabla^{0,1})$ to $(E,\nabla^{0,1}_{0})$ such that $\Phi_{0}\psi=\psi\Phi$.
But $\psi$ is non-zero, so we have the following two possibilities:
\begin{itemize}
    \item $\psi$ is an isomorphism, in which case it is in $\G^{c}_{0}$ and $(\nabla,\Phi)$ is in the same orbit as $(\nabla_{0},\Phi_{0})$, yielding the desired solution.
    \item The image of $\psi$ has generically rank $1$, going down to $0$ only at special points by upper-semicontinuity \cite[Proposition 1.3.2]{ati89}.
\end{itemize}

Assume then that we are in the second case.
Since $\psi$ is holomorphic, the saturation of $\psi(E)$ in $E$ defines a line bundle $L\subseteq E$ holomorphic with respect to $\nabla_{0}^{0,1}$.
But $\Phi_{0}\psi=\psi\Phi$, so $L$ is $\Phi_{0}$-invariant.
Indeed, it suffices to check this generically, so let $x\in M$ such that $L_{x}=\psi(E_{x})$ and let $v=\psi(v')\in L_{x}$.
Then $\phi_{0}(v)=\phi_{0}(\psi(v'))=\psi(\phi(v'))\in \psi(E_{x})$.
By \Cref{prop:general}, since $(\nabla_{0},\Phi_{0})$ is stable, this is impossible for $\Phi_{0}$ on a non-empty Zariski open subset.
So for a general $\Phi_{0}$ the limit $\psi$ had to be an isomorphism.

The general non-general case will follow from a further application of Uhlenbeck's theorem.

\bibliographystyle{alpha}
\bibliography{main}
\vfill

\end{document}
