\documentclass[A4paper, 12pt, british, reqno]{amsart}

\input{preamble} % Content described in README.md

% Math operators
\DeclareMathOperator{\Hom}{Hom}
\DeclareMathOperator{\id}{id}
\DeclareMathOperator{\GL}{GL}

% Other commands
\newcommand{\ot}{\otimes}
\newcommand{\op}{\oplus}
\newcommand{\dual}{^{\vee}}
\newcommand{\act}{\rotatebox[origin=c]{-120}{$\circlearrowright$}}
\newcommand{\Diff}{\mathbf{Diff}}
\newcommand{\Grp}{\mathbf{Grp}}

\begin{document}

\maketitle

\begin{abstract}
    In this talk we introduce the stability condition for Higgs bundles and prove the Hitchin--Kobayashi correspondence.
    The main result is \cite[Theorem 4.3]{hit87}.
    Relevant literature is \cite[\S 3 and \S 4]{hit87} and \cite[\S 2 and \S 3]{wen14}.
    Maybe we will also use \cite{wen16} every now and then.

    This talk is related to Tanuj's talk on \textit{Stable vector bundles}, for which the main reference is \cite{kob87}.
    Therefore we will also use \cite{kob87} as a main reference for generalities on complex vector bundles.
\end{abstract}

\tableofcontents

\section*{Notation and conventions}\label{sec:notation}

We usually follow the notation of \cite{hit87a}:
\begin{itemize}
    \item $M$: compact Riemann surface of genus $g$.
    \item $O\to M$: trivial line bundle.
    \item $K\to M$: canonical line bundle.
    \item More generally, $O_{X}$ and $K_{X}$ denote the trivial and canonical line bundles over a complex manifold $X$.
\end{itemize}
Every now and then we will also use some other standard notation, for instance:
\begin{itemize}
    \item For a vector bundle $E\to M$ we denote $\mu(E):=\deg{E}/\rk{E}$.
    \item Let $X$ be a complex manifold and $E\to X$ a (holomorphic/algebraic) vector bundle.
	Then we denote by $\E$ its sheaf of sections.
	The assignement $E\mapsto \E$ defines an equivalence of categories between vector bundles on $X$ and locally free sheaves of $\O_{X}$-modules of finite rank, which we will refer to simply as locally free sheaves.
	We can recover $E$ from $\E$ either using cocycles \cite[Lemma 4.8]{voi02} or by using the relative spectrum \cite[Exercise II.5.18]{har77}.
	Following the second approach we would write
	\[ E=\Spec_{X}(S(\E\dual))=:\V(\E\dual), \]
	where $S(-)$ denotes the symmetric algebra.
    \item $\O$ and $\w$ denote the trivial and canonical invertible sheaves on $M$.
	More generally, $\O_{X}$ and $\w_{X}$ denote the trivial and canonical invertible sheaves on a complex manifold $X$,
    \item Let $E$ be again a vector bundle on a complex manifold $X$.
	We will denote its projectivisation by $\P(E)$, which is obtained form $E$ without its zero section by quotienting out the fibre-wise multiplication by scalars.
	Taking again the algebraic approach we can write
	\[ \P(E)=\Proj_{X}(S(\E\dual))=:\P(\E\dual). \]
    \item Let $N$ be a smooth manifold and $E\to N$ a smooth vector bundle.
	Then we denote by $\X(N)$ the Lie algebra of smooth vector fields on $N$ and by $\Omega^{k}(N,E)$ the vector space of smooth $k$-differential forms with values in $E$, which can be thought of as smooth global sections of the vector bundle $\Hom(TN,E)$.
    \item Let $N$ be a smooth manifold equipped with an almost complex structure $I\colon TN\to TN$.
	Then we denote by $\Omega^{i,j}(N,E)$...
\end{itemize}

\bibliographystyle{alpha}
\bibliography{main}
\vfill

\end{document}


\section{Self-duality}

We consider $\R^{4}$ with its standard smooth structure \cite[Example 1.22]{lee13}.

\appendix

\section{Complex vector bundles, connections and curvature}

\begin{defn}[Complex vector bundle]
    Let $M\in \Diff$. 
    A \textit{complex vector bundle} on $M$ consists of a family $\{E_{x}\}_{x\in M}$ of complex vector spaces parametrized by $M$, together with a smooth manifold structure on $E:=\sqcup_{x\in M}E_{x}$ such that
    \begin{enumerate}[label=\roman*)]
	\item The projection map $\pi\colon E\to M$ taking $E_{x}$ to $x$ is smooth, and
	\item For every $x_{0}\in M$, there exists an open set $U$ in $M$ containing $x_{0}$ and a diffeomorphism
	    \[ \varphi_{U}\colon \pi^{-1}(U)\to U\times \C^{k} \]
	    taking the vector space $E_{x}$ isomorphically onto $\{x\}\times \C^{k}$ for each $x\in U$;
	    $\varphi_{U}$ is called a \textit{trivialization of $E$ over $U$}.
    \end{enumerate}
\end{defn}

\begin{rem}
    If $M$ is a complex manifold, we can also talk about \textit{holomorphic vector bundles}.
    These are complex vector bundles $\pi\colon E\to M$ together with a structure of complex manifold on $E$ such that we can find around each point a biholomorphic local trivialization $\varphi_{U}$.
\end{rem}

\begin{defn}[Complex differential forms]
    Let $M\in \Diff$ and let $T_{M}$ be its tangent bundle.
    Let $E\to M$ be a complex vector bundle on $M$.
    Then the bundle of \textit{complex $p$-forms with values in $E$} is defined as
    \[ \Omega_{M,\C}^{p}(E):=\bigwedge\nolimits^{p}\Hom_{M}(T_{M},E). \]
    A \textit{complex $p$-form with values in $E$} is then a smooth global section of $\Omega_{M,\C}^{p}(E)$.
    The $\C$-vector space of complex $p$-forms with values in $E$ will be denoted by $A^{p}(E)$.
\end{defn}

\begin{rem}
    In the particular case in which $E=M\times \C$ is the trivial complex line bundle on $M$, we simply talk about the bundle of complex $p$-forms on $M$, denoted $\Omega_{M,\C}^{p}$.
    Similarly, a smooth global section of $\Omega_{M,\C}^{p}$ will be simply called a complex $p$-form on $M$, and the $\C$-vector space of complex $p$-forms on $M$ will be denoted by $A^{p}$.
\end{rem}

\begin{defn}[Connection]
    Let $M\in \Diff$ and $E\to M$ a complex vector bundle.
    A \textit{connection} $D$ in $E$ is a $\C$-linear homomorphism
    \[ D\colon A^{0}(E)\to A^{1}(E) \]
    such that
    \[ D(f\sigma)=\sigma df +f\cdot D\sigma \]
    for $f\in A^{0}=C^{\infty}(M,\C)$ and $\sigma\in A^{0}(E)=\Gamma(M,E)$.
\end{defn}


\section{Principal bundles on smooth manifolds}

In this appendix we recall the basics of principal $G$-bundles on smooth manifolds, where $G$ is a Lie group.

\begin{defn}[Lie group]
    A \textit{Lie group} is a group object in the category $\Diff$ of smooth manifolds.
\end{defn}

\begin{rem}
    $G\in \Diff\Grp$ if and only if its underlying set is equipped with a group structure such that the map $G\times G\to G$ given by $(g,h)\to gh^{-1}$ is smooth \cite[Proposition 7.1]{lee13}.
\end{rem}

Recall that for $M\in \Diff$, the $\R$-vector space $\X(M)$ of smooth vector fields on $M$ forms a Lie algebra under the Lie bracket \cite[Proposition 8.28]{lee13}.

Let $M\in \Diff$ and $G\in \Diff\Grp$.
Then the projection $\pi\colon M\times G\to M$ has some nice properties, namely:
\begin{itemize}
    \item $G\act M\times G$ smoothly and fibrewise via $(x,g)\cdot h\mapsto (x,gh)$.
    \item For all $x\in M$, $G\act \pi^{-1}(x)$ induces $G\cong \{x\}\times G\cong \pi^{-1}(x)$.
\end{itemize}
The smooth manifold $M\times G$ over $M$ equipped with this right fibrewise action is called the \textit{trivial principal $G$-bundle} on $M$.
We can encode all this structure by saying that
\[ M\times G \in (G\text{-}\Diff)/M, \]
where we consider $M$ with the trivial $G$-action.

\begin{defn}[Principal bundle]
    Let $M\in \Diff$ and $G\in \Diff\Grp$.
    Consider $M\in G$-$\Diff$ with the trivial action.
    A \textit{principal $G$-bundle} on $M$ is an object $P\in (G\text{-}\Diff)/M$ which is trivial locally on $M$.
\end{defn}

\begin{exa}
    Let $M\in \Diff$ and $G:=\GL(n,\R)\in \Diff\Grp$.
    Then the \textit{frame bundle} of $M$, denoted $\GL(M)$, is the principal $G$-bundle whose fibra over $x\in M$ is the set of all frames for the tangent space $T_{x}M$.
\end{exa}

\begin{exa}
    Let $G\in \Diff$ and $H\subseteq G$ a closed subgroup.
    Then $G$ is a principal $H$-bundle over the left coset space $G/H$.
\end{exa}

Some nice properties in the topological category, which probably extend to the smooth category (check!):

\begin{prop}
    \begin{enumerate}[label=\roman*)]
	\item Any morphism of principal $G$-bundles is an isomorphism.
	\item A principal $G$-bundle is trivial if and only if it admits a section, where trivial means isomorphic to a trivial principal $G$-bundle.
    \end{enumerate}
\end{prop}


\bibliographystyle{alpha}
\bibliography{main}
\vspace{0.3cm}

\end{document}
