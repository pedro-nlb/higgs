\documentclass[A4paper, 12pt, british, reqno]{amsart}

%%%%%%%%%% CONTENTS OF PREAMBLE %%%%%%%%%%

% Packages ----------------- Line 10
% General things ----------- Line 72
% Font definitions --------- Line 83
% Theorem environments ----- Line 228
% Tikzcd ------------------- Line 400
% Author, title, etc ------- Line 420

%%%%%%%%%% PACKAGES %%%%%%%%%%

\usepackage[T1]{fontenc}
% This uses 8-bit font encoding (with 256 glyphs) instead of the default 7-bit font encoding (with 128 glyphs). For example, with this option ö is a single glyph in the font, whereas on the 7-bit font encoding the font ö is made by adding an accent to the existing glyph o. A bad consequence of not using this package is that you cannot properly copy-paste such words form the output pdf file. Also, for some reason, funny stuff happens with |, < and > in text.

% Some people suggest to load fontenc before inputenc, most agree that it does not matter.

\usepackage[utf8]{inputenc}
% When you type ä in an editor set up for utf8, the machine stores the character number 228. When TeX reads the file it finds the character number 228 and the macros of inputenc transform this into \"a. Finally fontenc does its thing and transforms this into the command print character 228 (otherwise the two things would be printed separatedly as explained in fontenc).

\usepackage[UKenglish]{babel}
% To manage culturally determined typographical and similar rules, in this case for british english. Some people suggest to load babel after fontenc to avoid warnings, although most agree that it does not matter.

\usepackage{mathtools}
% Loads the amsmath package (\usepackage{amsmath}: miscellaneous improvements such as the commands \DeclareMathOperator and \text). It fixes some quirks it has and adds some useful settings, symbols and environments. It improves the aesthetics as well.

\usepackage{amssymb}
% Extended symbol collection, e.g. \Cap and \Cup. More importantly: the \mathbb command! It loads the amsfonts package (\usepackage{amsfonts}: fraktur letters, bold Greek letters...), so we do not need to include it in the preamble anymore.

\usepackage{mathrsfs}
% Font package (only supports upper case letters).

\usepackage{enumitem}
% To control the layout of enumerate, itemize and description. It supersedes the enumerate package.

\usepackage{tikz-cd}
% To draw commutative diagrams.
\usetikzlibrary{decorations.markings}
% For open and closed immersions.

\usepackage{graphicx}
% An extension of the graphics package, with optional arguments for the \includegraphics command.

\usepackage{todonotes}
% To write to do notes use the command \todo.

\usepackage{xcolor}
% To write in colors.

\usepackage{mathdots}
% To draw diagonal dots.

\usepackage{marginnote}
% To write on margins.

\usepackage{manfnt}
% To draw dangerous bent symbol.

\usepackage{libertine,libertinust1math}

\usepackage[mathscr]{euscript}
% For mathscr font family.

\usepackage{float}
% Improved interface for floating objects such as figures and tables, introducing for example the H modifier to force the position of a float in the page or the boxed float. Should be loaded before hyperref.

\usepackage[backref=page]{hyperref}
% To handle cross-referencing and produce hypertext links in the document. It should be loaded last (with few exceptions), because it redefines many LaTeX commands.
% The backref option inserts links on each bibliography entry to the pages in which the citation was used.
%% The hidelinks option removes colors and boxes around links, but the links remain clickable. On firefox the links are even highlighted when the mouse pointer passes over them.
\renewcommand{\backref}[1]{$\uparrow$~#1}
% Adds an upwards arrow before referencing to the pages in which the citations appear.

\usepackage[noabbrev]{cleveref}
% Enhances cross-referencing features, e.g. to reference to a theorem and automatically include the word theorem.
% No abbreviature option to write figure instead of fig. etc.

%%%%%%%%%% GENERAL THINGS %%%%%%%%%%

% Limit table of contents to section titles
\setcounter{tocdepth}{1}

% Sloppy formatting -- often looks better
\sloppy

% To make long aligned equations look better
\allowdisplaybreaks

%%%%%%%%%% FONT DEFINITIONS %%%%%%%%%%

% Sheaves
\newcommand{\sA}{\mathscr{A}}
\newcommand{\sB}{\mathscr{B}}
\newcommand{\sC}{\mathscr{C}}
\newcommand{\sD}{\mathscr{D}}
\newcommand{\E}{\mathscr{E}} % Exception (Vector bundles)
\newcommand{\F}{\mathscr{F}} % Exception (Coherent sheaves)
\newcommand{\G}{\mathscr{G}} % Exception (Coherent sheaves)
\newcommand{\sH}{\mathscr{H}}
\newcommand{\sHom}{\mathscr{H}\hspace{-0.2mm} om} % Exception (Hom-sheaf)
\newcommand{\sEnd}{\mathscr{E}\hspace{-0.2mm} nd} % Exception (End-sheaf)
\newcommand{\I}{\mathscr{I}} % Exception (Ideal sheaves)
\newcommand{\sJ}{\mathscr{J}}
\newcommand{\sK}{\mathscr{K}}
\renewcommand{\L}{\mathscr{L}} % Exception (Line bundles)
\newcommand{\M}{\mathscr{M}} % Exception (Line bundles)
\newcommand{\sN}{\mathscr{N}}
\renewcommand{\O}{\mathscr{O}} % Exception (Structure sheaf)
\newcommand{\sP}{\mathscr{P}}
\newcommand{\sQ}{\mathscr{Q}}
\newcommand{\sR}{\mathscr{R}}
\newcommand{\sS}{\mathscr{S}}
\newcommand{\sT}{\mathscr{T}}
\newcommand{\sU}{\mathscr{U}}
\newcommand{\sV}{\mathscr{V}}
\newcommand{\sW}{\mathscr{W}}
\newcommand{\w}{\omega} % Addition (Canonical sheaf)
\newcommand{\sX}{\mathscr{X}}
\newcommand{\sY}{\mathscr{Y}}
\newcommand{\sZ}{\mathscr{Z}}

% Mathcal fonts
\newcommand{\calA}{\mathcal{A}}
\newcommand{\calB}{\mathcal{B}}
\newcommand{\calC}{\mathcal{C}}
\newcommand{\calD}{\mathcal{D}}
\newcommand{\calE}{\mathcal{E}}
\newcommand{\calF}{\mathcal{F}}
\newcommand{\calG}{\mathcal{G}}
\newcommand{\calH}{\mathcal{H}}
\newcommand{\calI}{\mathcal{I}}
\newcommand{\calJ}{\mathcal{J}}
\newcommand{\calK}{\mathcal{K}}
\newcommand{\calL}{\mathcal{L}}
\newcommand{\calM}{\mathcal{M}}
\newcommand{\calN}{\mathcal{N}}
\newcommand{\calO}{\mathcal{O}}
\newcommand{\calP}{\mathcal{P}}
\newcommand{\calQ}{\mathcal{Q}}
\newcommand{\calR}{\mathcal{R}}
\newcommand{\calS}{\mathcal{S}}
\newcommand{\calT}{\mathcal{T}}
\newcommand{\U}{\mathcal{U}} % Exception (Open covers)
\newcommand{\calV}{\mathcal{V}}
\newcommand{\calW}{\mathcal{W}}
\newcommand{\X}{\mathcal{X}} % Exception (Families of varieties)
\newcommand{\Y}{\mathcal{Y}} % Exception (Families of varieties)
\newcommand{\calZ}{\mathcal{Z}}

% Blackboard Bold Symbols
\newcommand{\A}{\mathbb{A}} % Exception (Affine space)
\newcommand{\bbB}{\mathbb{B}}
\newcommand{\C}{\mathbb{C}} % Exception (Complex numbers)
\newcommand{\bbD}{\mathbb{D}}
\newcommand{\bbE}{\mathbb{E}}
\newcommand{\bbF}{\mathbb{F}}
\newcommand{\bbG}{\mathbb{G}}
\newcommand{\Gm}{\mathbb{G}_{\mathrm{m}}} % Addition (Punctured affine line)
\newcommand{\bbH}{\mathbb{H}}
\newcommand{\bbI}{\mathbb{I}}
\newcommand{\bbJ}{\mathbb{J}}
\newcommand{\bbK}{\mathbb{K}}
\renewcommand{\k}{\Bbbk} % Addition (Field)
\newcommand{\bbL}{\mathbb{L}}
\newcommand{\bbM}{\mathbb{M}}
\newcommand{\N}{\mathbb{N}} % Exception (Natural numbers)
\newcommand{\bbO}{\mathbb{O}}
\renewcommand{\P}{\mathbb{P}} % Exception (Projective space)
\newcommand{\Q}{\mathbb{Q}} % Exception (Rational numbers)
\newcommand{\R}{\mathbb{R}} % Exception (Real numbers)
\newcommand{\bbS}{\mathbb{S}}
\newcommand{\bbT}{\mathbb{T}}
\newcommand{\bbU}{\mathbb{U}}
\newcommand{\V}{\mathbb{V}} % Exception (Geometric vector bundle)
\newcommand{\bbW}{\mathbb{W}}
\newcommand{\bbX}{\mathbb{X}}
\newcommand{\bbY}{\mathbb{Y}}
\newcommand{\Z}{\mathbb{Z}} % Exception (Integers)

% Boldfont (categories)
\newcommand{\bfA}{\mathbf{A}}
\newcommand{\Ab}{\mathbf{Ab}}
\newcommand{\bfB}{\mathbf{B}}
\newcommand{\bfC}{\mathbf{C}}
\newcommand{\Cat}{\mathbf{Cat}} % Addition (Categories)
\newcommand{\Coh}{\mathbf{Coh}} % Addition (Coherent sheaves)
\newcommand{\D}{\mathbf{D}} % Exception (Derived category)
\newcommand{\Db}{\mathbf{D}^{\mathrm{b}}} % Addition (Bounded derived category)
\newcommand{\bfE}{\mathbf{E}}
\newcommand{\bfF}{\mathbf{F}}
\newcommand{\bfG}{\mathbf{G}}
\newcommand{\bfH}{\mathbf{H}}
\newcommand{\bfI}{\mathbf{I}}
\newcommand{\bfJ}{\mathbf{J}}
\newcommand{\K}{\mathbf{K}} % Exception (Homotopy category)
\newcommand{\bfL}{\mathbf{L}}
\newcommand{\bfM}{\mathbf{M}}
\newcommand{\Mod}{\mathbf{Mod}} % Addition (Modules)
\newcommand{\bfN}{\mathbf{N}}
\newcommand{\bfO}{\mathbf{O}}
\newcommand{\bfP}{\mathbf{P}}
\newcommand{\PSh}{\mathbf{PSh}} % Addition (Presheaves)
\newcommand{\bfQ}{\mathbf{Q}}
\newcommand{\QCoh}{\mathbf{QCoh}} % Addition (Quasi-coherent sheaves)
\newcommand{\bfR}{\mathbf{R}}
\newcommand{\bfS}{\mathbf{S}}
\newcommand{\Set}{\mathbf{Set}} % Addition (Sets)
\newcommand{\Sh}{\mathbf{Sh}} % Addition (Sheaves)
\newcommand{\bfT}{\mathbf{T}}
\newcommand{\Top}{\mathbf{Top}} % Addition (Topological spaces)
\newcommand{\bfU}{\mathbf{U}}
\newcommand{\bfV}{\mathbf{V}}
\renewcommand{\Vec}{\mathbf{Vec}} % Addition (Vector bundles)
\newcommand{\bfW}{\mathbf{W}}
\newcommand{\bfX}{\mathbf{X}}
\newcommand{\bfY}{\mathbf{Y}}
\newcommand{\bfZ}{\mathbf{Z}}

% Mathfrak for ideals
\renewcommand{\a}{\mathfrak{a}}
\renewcommand{\b}{\mathfrak{b}}
\renewcommand{\c}{\mathfrak{c}}
\renewcommand{\d}{\mathfrak{d}}
\newcommand{\e}{\mathfrak{e}}
\newcommand{\m}{\mathfrak{m}}
\newcommand{\n}{\mathfrak{n}}

%%%%%%%%%% THEOREM ENVIRONMENTS %%%%%%%%%%

% Custom colors for theorem styles below
\definecolor{darkgreen}{RGB}{0,75,0}
\definecolor{darkblue}{RGB}{0,0,75}
\definecolor{darkred}{RGB}{75,0,0}
\definecolor{linkred}{rgb}{0.7,0.2,0.2}
\definecolor{linkblue}{rgb}{0,0.2,0.6}

% Custom theorem styles (empty fields take default values)
\newtheoremstyle{darkgreentheorem}% name of the style
{}% measure of space to leave above the theorem. E.g.: 3pt
{}% measure of space to leave below the theorem. E.g.: 3pt
{\itshape}% name of font to use in the body of the theorem
{}% measure of space to indent
{\color{darkgreen}\bfseries}% name of head font
{.}% punctuation between head and body
{ }% space after theorem head; " " = normal interword space
{}% Manually specify head
\newtheoremstyle{darkbluedefinition}
{}{}{}{}{\color{darkblue}\bfseries}{.}{ }{}
\newtheoremstyle{darkredexample}
{}{}{}{}{\color{darkred}\bfseries}{.}{ }{}

% Numbered theorems
\theoremstyle{plain}
% \theoremstyle{darkgreentheorem}
\newtheorem{thm}{Theorem}
\newtheorem{lm}[thm]{Lemma}
\newtheorem{prop}[thm]{Proposition}
\newtheorem{cor}[thm]{Corollary}
\newtheorem{conj}[thm]{Conjecture}
\newtheorem{fact}[thm]{Fact}
\theoremstyle{definition}
% \theoremstyle{darkbluedefinition}
\newtheorem{defn}[thm]{Definition}
\newtheorem{defnlm}[thm]{Definition-Lemma}
% \theoremstyle{darkredexample}
\newtheorem{exa}[thm]{Example}
\theoremstyle{remark}
\newtheorem{rem}[thm]{Remark}
\newtheorem{nota}[thm]{Notation}
\newtheorem{q}[thm]{Question}
\newtheorem{exe}[thm]{Exercise}

% Custom numbered theorems
\theoremstyle{plain}
% \theoremstyle{darkgreentheorem}
\newtheorem{innercustomthm}{Theorem}
\newenvironment{cthm}[1]
    {\renewcommand\theinnercustomthm{#1}\innercustomthm}
    {\endinnercustomthm}
\newtheorem{innercustomlm}{Lemma}
\newenvironment{clm}[1]
    {\renewcommand\theinnercustomlm{#1}\innercustomlm}
    {\endinnercustomlm}
\newtheorem{innercustomprop}{Proposition}
\newenvironment{cprop}[1]
    {\renewcommand\theinnercustomprop{#1}\innercustomprop}
    {\endinnercustomprop}
\newtheorem{innercustomcor}{Corollary}
\newenvironment{ccor}[1]
    {\renewcommand\theinnercustomcor{#1}\innercustomcor}
    {\endinnercustomcor}
\newtheorem{innercustomconj}{Conjecture}
\newenvironment{cconj}[1]
    {\renewcommand\theinnercustomconj{#1}\innercustomconj}
    {\endinnercustomconj}
\newtheorem{innercustomfact}{Fact}
\newenvironment{cfact}[1]
    {\renewcommand\theinnercustomfact{#1}\innercustomfact}
    {\endinnercustomfact}
% Definitions
\theoremstyle{definition}
% \theoremstyle{darkbluedefinition}
\newtheorem{innercustomdefn}{Definition}
\newenvironment{cdefn}[1]
    {\renewcommand\theinnercustomdefn{#1}\innercustomdefn}
    {\endinnercustomdefn}
\newtheorem{innercustomdefnlm}{Definition-Lemma}
\newenvironment{cdefnlm}[1]
    {\renewcommand\theinnercustomdefnlm{#1}\innercustomdefnlm}
    {\endinnercustomdefnlm}
\newtheorem{innercustomexe}{Exercise}
\newenvironment{cexe}[1]
    {\renewcommand\theinnercustomexe{#1}\innercustomexe}
    {\endinnercustomexe}
% \theoremstyle{darkredexample}
\newtheorem{innercustomexa}{Example}
\newenvironment{cexa}[1]
    {\renewcommand\theinnercustomexa{#1}\innercustomexa}
    {\endinnercustomexa}
\theoremstyle{remark}
\newtheorem{innercustomrem}{Remark}
\newenvironment{crem}[1]
    {\renewcommand\theinnercustomrem{#1}\innercustomrem}
    {\endinnercustomrem}
\newtheorem{innercustomnota}{Notation}
\newenvironment{cnota}[1]
    {\renewcommand\theinnercustomnota{#1}\innercustomnota}
    {\endinnercustomnota}
\newtheorem{innercustomq}{Question}
\newenvironment{cq}[1]
    {\renewcommand\theinnercustomq{#1}\innercustomq}
    {\endinnercustomq}

% Unnumbered theorems
\theoremstyle{plain}
% \theoremstyle{darkgreentheorem}
\newtheorem*{uthm}{Theorem}
\newtheorem*{ulm}{Lemma}
\newtheorem*{uprop}{Proposition}
\newtheorem*{ucor}{Corollary}
\newtheorem*{uconj}{Conjecture}
\newtheorem*{ufact}{Fact}
\theoremstyle{definition}
% \theoremstyle{darkbluedefinition}
\newtheorem*{udefn}{Definition}
\newtheorem*{udefnlm}{Definition-Lemma}
% \theoremstyle{darkredexample}
\newtheorem*{uexa}{Example}
\theoremstyle{remark}
\newtheorem*{urem}{Remark}
\newtheorem*{unota}{Notation}
\newtheorem*{uq}{Question}
\newtheorem*{uexe}{Exercise}

% Cross-referencing
\crefname{thm}{theorem}{theorems}
\Crefname{thm}{Theorem}{Theorems}
\crefname{lm}{lemma}{lemmas}
\Crefname{lm}{Lemma}{Lemmas}
\crefname{prop}{proposition}{propositions}
\Crefname{prop}{Proposition}{Propositions}
\crefname{cor}{corollary}{corollaries}
\Crefname{cor}{Corollary}{Corollaries}
\crefname{conj}{conjecture}{conjectures}
\Crefname{conj}{Conjecture}{Conjectures}
\crefname{fact}{fact}{facts}
\Crefname{fact}{Fact}{Facts}
\crefname{defn}{definition}{definitions}
\Crefname{defn}{Definition}{Definitions}
\crefname{defnlm}{defition-lemma}{definition-lemmas}
\Crefname{defnlm}{Definition-Lemma}{Definition-Lemmas}
\crefname{exa}{example}{examples}
\Crefname{exa}{Example}{Examples}
\crefname{rem}{remark}{remarks}
\Crefname{rem}{Remark}{Remarks}
\crefname{nota}{notation}{notations}
\Crefname{nota}{Notation}{Notations}
\crefname{q}{question}{questions}
\Crefname{q}{Question}{Questions}
\crefname{exe}{exercise}{exercises}
\Crefname{exe}{Exercise}{Exercises}
% More cross-referencing
\crefname{cthm}{theorem}{theorems}
\Crefname{cthm}{Theorem}{Theorems}
\crefname{clm}{lemma}{lemmas}
\Crefname{clm}{Lemma}{Lemmas}
\crefname{cprop}{proposition}{propositions}
\Crefname{cprop}{Proposition}{Propositions}
\crefname{ccor}{corollary}{corollaries}
\Crefname{ccor}{Corollary}{Corollaries}
\crefname{cconj}{conjecture}{conjectures}
\Crefname{cconj}{Conjecture}{Conjectures}
\crefname{cfact}{fact}{facts}
\Crefname{cfact}{Fact}{Facts}
\crefname{cdefn}{definition}{definitions}
\Crefname{cdefn}{Definition}{Definitions}
\crefname{cdefnlm}{definition-lemma}{definition-lemmas}
\Crefname{cdefnlm}{Definition-Lemma}{Definition-Lemmas}
\crefname{cexa}{example}{examples}
\Crefname{cexa}{Example}{Examples}
\crefname{crem}{remark}{remarks}
\Crefname{crem}{Remark}{Remarks}
\crefname{cnota}{notation}{notations}
\Crefname{cnota}{Notation}{Notations}
\crefname{cq}{question}{questions}
\Crefname{cq}{Question}{Questions}
\crefname{cexe}{exercise}{exercises}
\Crefname{cexe}{Exercise}{Exercises}

%%%%%%%%%% TIKZCD %%%%%%%%%%

% Open and closed immersion arrows.
\makeatletter
\tikzcdset{
open/.code={\tikzcdset{hook, circled};},
closed/.code={\tikzcdset{hook, slashed};},
circled/.code={\tikzcdset{markwith={\draw (0,0) circle (.375ex);}};},
slashed/.code={\tikzcdset{markwith={\draw[-] (-.4ex,-.4ex) -- (.4ex,.4ex);}};},
markwith/.code={
\pgfutil@ifundefined{tikz@library@decorations.markings@loaded}%
{\pgfutil@packageerror{tikz-cd}{You need to say %
\string\usetikzlibrary{decorations.markings} to use arrow with markings}{}}{}%
\pgfkeysalso{/tikz/postaction={/tikz/decorate,
/tikz/decoration={
markings,
mark = at position 0.5 with
{#1}}}}},
}
\makeatother

%%%%%%%%%% AUTHOR, TITLE, ETC %%%%%%%%%%

% Author info
\author{Pedro Núñez}
\address{Pedro Núñez, Mathematisches Institut, Albert-Ludwigs-Universität Freiburg, Ernst-Zermelo-Straße 1, 79104 Freiburg im Breisgau, Germany}
\email{\href{mailto:pedro.nunez@math.uni-freiburg.de}{pedro.nunez@math.uni-freiburg.de}}
\urladdr{\href{https://home.mathematik.uni-freiburg.de/nunez/}{https://home.mathematik.uni-freiburg.de/nunez}}
\thanks{Supported by the DFG-Graduiertenkolleg GK1821 ``Cohomological Methods in Geometry'' at the University of Freiburg.}

% Content details
%\keywords{...}
%\subjclass[...]{...}
\title[Higgs Bundles --- Existence of Solutions]{Higgs Bundles --- Existence of Solutions}
\date{15 July 2020}

% Links and pdf options
\makeatletter
\hypersetup{
  pdfauthor={\authors},
  pdftitle={\@title},
  %pdfsubject={\@subjclass},
  %pdfkeywords={\@keywords},
  pdfstartview={Fit},
  pdfpagelayout={TwoColumnRight},
  pdfpagemode={UseOutlines},
  bookmarks,
  colorlinks,
  linkcolor=linkblue,
  citecolor=linkred,
  urlcolor=linkred}
\makeatother
 % Content described in README.md

% Math operators
\DeclareMathOperator{\Hom}{Hom}
\DeclareMathOperator{\End}{End}
\DeclareMathOperator{\Sym}{Sym}
\DeclareMathOperator{\Spec}{Spec}
\DeclareMathOperator{\Proj}{Proj}
\DeclareMathOperator{\Pic}{Pic}
\DeclareMathOperator{\rk}{rk}
\DeclareMathOperator{\id}{id}
\DeclareMathOperator{\ad}{ad}
\DeclareMathOperator{\GL}{GL}
\let\div\relax
\DeclareMathOperator{\div}{div}
\DeclareMathOperator{\SO}{SO}
\let\hom\relax
\DeclareMathOperator{\hom}{hom}
\let\Im\relax
\DeclareMathOperator{\Im}{Im}
\DeclareMathOperator{\Ker}{Ker}

% Other commands
\newcommand{\ot}{\otimes}
\newcommand{\op}{\oplus}
\newcommand{\dual}{^{*}}
\newcommand{\act}{\rotatebox[origin=c]{-120}{$\circlearrowright$}}
\newcommand{\Diff}{\mathbf{Diff}}
\newcommand{\Grp}{\mathbf{Grp}}

\begin{document}

\maketitle

\begin{abstract}
    In this talk we introduce the stability condition for Higgs bundles and prove the Hitchin--Kobayashi correspondence.
    The main result is \cite[Theorem 4.3]{hit87a}.
    Relevant literature is \cite[\S 3 and \S 4]{hit87a} and \cite[\S 2 and \S 3]{wen14}.
    Maybe we will also use \cite{wen16} every now and then.

    This talk is related to Tanuj's talk on \textit{Stable vector bundles}, for which the main reference is \cite{kob87}.
    Therefore we will also use \cite{kob87} as a default reference for generalities on complex vector bundles.
\end{abstract}

\tableofcontents

\section*{Notation and conventions}

We usually follow the notation of \cite{hit87a}:
\begin{itemize}
    \item $M$: compact Riemann surface of genus $g$.
    \item $O\to M$: trivial line bundle.
    \item $K\to M$: canonical line bundle.
    \item More generally, $O_{X}$ and $K_{X}$ denote the trivial and canonical line bundles over a complex manifold $X$.
\end{itemize}
Every now and then we will also use some other standard notation, for instance:
\begin{itemize}
    \item For a vector bundle $E\to M$ we denote $\mu(E):=\deg{E}/\rk{E}$.
    \item Let $X$ be a complex manifold and $E\to X$ a (holomorphic/algebraic) vector bundle.
	Then we denote by $\E$ its sheaf of sections.
	The assignement $E\mapsto \E$ defines an equivalence of categories between vector bundles on $X$ and locally free sheaves of $\O_{X}$-modules of finite rank, which we will refer to simply as locally free sheaves.
	We can recover $E$ from $\E$ either using cocycles \cite[Lemma 4.8]{voi02} or by using the relative spectrum \cite[Exercise II.5.18]{har77}.
	Following the second approach we would write
	\[ E=\Spec_{X}(S(\E\dual))=:\V(\E\dual), \]
	where $S(-)$ denotes the symmetric algebra.
    \item $\O$ and $\w$ denote the trivial and canonical invertible sheaves on $M$.
	More generally, $\O_{X}$ and $\w_{X}$ denote the trivial and canonical invertible sheaves on a complex manifold $X$,
    \item Let $E$ be again a vector bundle on a complex manifold $X$.
	We will denote its projectivisation by $\P(E)$, which is obtained form $E$ without its zero section by quotienting out the fibre-wise multiplication by scalars.
	Taking again the algebraic approach we can write
	\[ \P(E)=\Proj_{X}(S(\E\dual))=:\P(\E\dual). \]
\end{itemize}

\section{Stability}

%[explain that the self-duality equations were originally talking about some principal $G$-bundle on space time $\R^{4}$, but after a series of simplifications \cite[\S 1]{hit87a} we ended up talking about rank 2 complex vector bundles on a compact Riemann surface]

%[picture of moduli space of holomorphic rank 2 vector bundles]
%
%[Upshot: a rank 2 holomorphic vector bundle doesn't quite determine a solution to the self-duality equation, but almost! Vector bundle + tangent vector in its moduli space does.]
%
%[Use fact from moduli space + Serre duality to translate this tangent vector into a Higgs field]
%
%From now on let $M$ be a compcat Riemann surface (of genus $g\geqslant 2$) and $K\to M$ its canonical line bundle.

\begin{defn}[Higgs bundle]
    A \textit{Higgs bundle} on $M$ is a pair $(E,\Phi)$, where $E\to M$ is a rank $2$ vector bundle and $\Phi$ is a global section of $\End{E}\ot K$, called a \textit{Higgs field} on $E$.
\end{defn}

\begin{rem}
    Using the canonical isomorphisms
    \[ H^{0}(M,\sEnd(\E)\ot \w)\cong \Hom(\O,\E\dual \ot \E\ot \w)\cong \Hom(\E,\E\ot \w) \]
    we may identify $\Phi$ with a morphism
    \[ \Phi\colon E\to E\ot K. \]
\end{rem}

%\begin{rem}
%    We are mainly interested in the subbundle of traceless endomorphisms, denoted $\End_{0}{V}$.
%\end{rem}

%\begin{exa}
%    Assume $g\geqslant 2$.
%    Then $\deg{K}=2g-2>0$, so we can find a line bundle $K^{\frac{1}{2}}$ such that $K^{\frac{1}{2}}\ot K^{\frac{1}{2}}\cong K$.
%    Let $E=K^{\frac{1}{2}}\op K^{-\frac{1}{2}}$, where $K^{-\frac{1}{2}}=(K^{\frac{1}{2}})^{-1}$.
%    We consider the Higgs field $\Phi_{w}\colon K^{\frac{1}{2}}\op K^{-\frac{1}{2}}\to (K^{\frac{1}{2}}\op K^{-\frac{1}{2}})\ot K$ given by a matrix
%    \[
%	\begin{pmatrix}
%	    0 & w \\
%	    1 & 0
%	\end{pmatrix},
%    \]
%    where $w\in \Hom(K^{-\frac{1}{2}},K^{\frac{1}{2}}\ot K)\cong H^{0}(M,K^{2})$ can be regarded as a quadratic differential.
%\end{exa}

\begin{defn}[Stability]
A Higgs bundle $(E,\Phi)$ is said to be \textit{stable} if for every $\Phi$-invariant\footnote{Meaning that $\Phi(L)\subseteq L\ot K$.} line bundle $L\subseteq E$ we have $\mu(L)<\mu(E)$.
\end{defn}

\begin{rem}
    $(E,0)$ is stable if and only if $E$ is stable in the usual sense.
\end{rem}

\begin{exe}
    There are no stable Higgs bundles on $\P^{1}$.
    [Hint: Grothedieck's theorem allows us to write $\Phi$ as a matrix.
    What can we say about each entry?
    Solution in \cite{hit87a}]
\end{exe}

%\begin{exa}
%    Assume $g\geqslant 2$ and consier $E=K^{\frac{1}{2}}\op K^{-\frac{1}{2}}$ again.
%    Then $\Phi_{0}$ is stable, because $K^{-\frac{1}{2}}$ is the only $\Phi_{0}$-invariant line bundle and
%    \[ \deg{K^{-\frac{1}{2}}}=1-g<0=\frac{\deg{E}}{2}. \]
%\end{exa}

\begin{prop}
    Let $(E_{1},\Phi_{1})$ and $(E_{2},\Phi_{2})$ be stable pairs with $\Lambda^{2}E_{1}\cong \Lambda^{2}E_{2}$.
    Let $\Psi\colon E_{1}\to E_{2}$ be a non-zero morphism such that $(\Psi\ot \id_{\w}) \circ \Phi_{1}=\Phi_{2}\circ \Psi$.
    Then $\Psi$ is an isomorphism.
    \begin{proof}
	We prove the result by contradiction.
	Suppose $\Psi\colon \E_{1}\to \E_{2}$ is not an isomorphism.
	Since we are on an algebraic curve and $\Ker(\Psi)$ is a torsion-free coherent sheaf, it is a locally free sheaf of rank $1$ or $2$.
	Hence we can find an invertible sheaf $\L_{1}\subseteq \Ker(\Psi)\subseteq \E_{1}$.
	Similarly, we can find an invertible sheaf $\Im(\Psi)\subseteq \L_{2}\subseteq \E_{2}$.
	Since $\Psi\circ \Phi_{1}=\Phi_{2}\circ \Psi$, $\L_{i}$ is $\Phi_{i}$-invariant for each $i\in \{1,2\}$.
    \end{proof}
\end{prop}

\begin{prop}
    Assume $g\geqslant 2$ and let $E\to M$ be a rank $2$ vector bundle.
    Then there exists Higgs field $\Phi$ on $E$ such that $(E,\Phi)$ is stable if and only if there exists a dense Zariski open subset $U\subseteq H^{0}(M,\sEnd(\E)\ot \w)$ such that all $\Phi'\in U$ have the property that no line bundle $L\subseteq E$ is $\Phi'$-invariant.
    \begin{proof}
	We define the following sets of rank $2$ vector bundles on $M$:
	\begin{itemize}
	    \item $\bfS:=\{ E\in \Vec_{2}(M)\mid \exists \Phi \text{ such that }(E,\Phi)\text{ is stable}\}$.
	    \item $\bfA:=\{ E\in \Vec_{2}(M)\mid \text{a general }\Phi\text{ has no invariant }L \}$.
	    \item $\bfB:=\{ E\in \Vec_{2}(M)\mid \exists L\text{ invariant for all }\Phi\}$.
	\end{itemize}

	Our goal is to show that $\bfS=\bfA$.
	If $\Phi$ has no invariant $L$, then $(E,\Phi)$ is automatically stable.
	Hence $\bfA\subseteq \bfS$.
	The plan to show the other inclusion is to see that
	\[ \Vec_{2}(M)=\bfA\sqcup \bfB \]
	and that $\bfB\subseteq \Vec_{2}(M)\setminus \bfS$.
	
	Let us start by showing that $\Vec_{2}(M)=\bfA\sqcup \bfB$.

	Let $p\colon \P(E)\to M$ be the projectivisation of our rank 2 vector bundle, which is a ruled surface in the sense of \cite[\S V.2]{har77}.
	Let $O(-1)$ denote the tautological line bundle on $\P(E)$, whose fibre over $[v]\in \P(E)$ is the line $\{\lambda v \mid \lambda\in \C\}\subseteq E_{p([v])}$.
	Let $O(l):=O(1)^{\ot l}$ for all $l\in \Z$, and if $F\to \P(E)$ is another vector bundle, denote by $F(l)$ the tensor product $F\ot O(l)$.
	We have then $p_{*}\O(l)=S^{l}(\E\dual)$ for all $l\geqslant 0$ \cite[Exercise III.8.4]{har77}.

	Let $x\in M$.
	Then every endomorphism $A\in \End(E_{x})$ defines a quadratic map $E_{x}\to \Lambda^{2}E_{x}$ sending $v$ to $Av\wedge v$.
	Such a quadratic map can be naturally regarded as a degree $2$ homogeneous polynomial on the coordinates of $e$ with coefficients in $\Lambda^{2}E_{x}$.
	Hence we have a morphism $\End(E)\to S^{2}E\dual \ot \Lambda^{2}E$, which vanishes precisely along the trivial line subbundle of $\End(E)$ consisting over each fibre of scalar multiples of the identity.
	Sending $A\mapsto A-\frac{\operatorname{tr}(A)}{2}\id_{E_{x}}$ on each fibre allows us to identify $\End_{0}(E)$ as the quotient of $\End(E)$ by this trivial line subbundle, so we obtain an injective morphism $\End_{0}(E)\to S^{2}E\dual \ot \Lambda^{2}E$.
	Counting ranks we see that we have in fact an isomorphism $\End_{0}(E)\cong S^{2}E\dual \ot \Lambda^{2}E$, and therefore
	\[ \sEnd_{0}(\E)\ot \w\cong p_{*}\O(2)\ot \w\ot \Lambda^{2}\E. \]
	By the projection formula, $\sEnd_{0}(\E)\ot \w\cong p_{*}(p^{*}(\w\ot \Lambda^{2}\E)(2))$.
	Therefore we have an isomorphism
	\[ \alpha \colon H^{0}(M,\sEnd_{0}(\E)\ot \w)\cong H^{0}(\P(E),p^{*}(\w\ot \Lambda^{2}\E)(2)). \]

	Let now $\Phi$ be a traceless Higgs field on $E$, and assume it is non-zero.
	By construction, a non-zero vector $v\in E$ is an eigenvector of the twisted endomorphism over the corresponding fibre if and only if the section $\alpha(\Phi)$ vanishes at the point $[v]\in \P(E)$, i.e.~if and only if $[v]$ is in the divisor of zeros of the global section $\alpha(\Phi)$, which we denote $\div(\alpha(\Phi))$.
	Let $L\subseteq E$ be a $\Phi$-invariant subbundle, which defines a section of $p\colon \P(E)\to M$ by functoriality of projectivisation on injective morphisms of vector bundles:
	\begin{center}
	    \begin{tikzcd}
		\P(L)\arrow[equal]{d}\arrow{r}{\sigma} & \P(E)\arrow{dl}{p} \\
		M &
	    \end{tikzcd}
	\end{center}
	Being $\Phi$-invariant means precisely that $\sigma(M)\subseteq \div(s(\Phi))$.
	But then any non-zero $v\in L$ is a non-zero eigenvector corresponding to some eigenvalue of the endomorphism over the corresponding fibre.
	Since $\Phi$ is traceless and non-zero, the other eigenvalue must be different, and there must be some non-zero eigenvector outside of $L$, call it $v'\in V$.
	Since $v'$ is a non-zero eigenvector, $[v']\in \div(\alpha(\Phi))$.
	And since $v'\not\in L$, $[v']\not\in \sigma(M)$.
	Therefore $\sigma(M)$ is a proper irreducible component of the divisor $\div(\alpha(\Phi))$.
	So if $\div(\alpha(\Phi))$ is irreducible, then no line bundle $L\subseteq V$ is $\Phi$-invariant and $(E,\Phi)$ is automatically stable.


	Next we give a lower bound for the dimension of the linear system $|p^{*}(\w\ot\Lambda^{2}\E)(2)|$, which is one less than the dimension of the vector space $H^{0}(\P(E),p^{*}(\w\ot\Lambda^{2}\E)(2))$.
	Using the previous isomorphism it suffices to gain control over the dimension of the global sections of $\sEnd_{0}(\E)\ot \w$ on $M$, for which we can apply Hirzebruch--Riemann--Roch \cite[Theorem A.4.1]{har77}.
	From \cite[Example A.4.1.1]{har77} we get
	\[ \operatorname{td}(\w\dual)=1-\frac{c_{1}(\w)}{2}. \]
	Using the short exact sequence used earlier
	\[ 0\to \O\to \sEnd(\E) \to \sEnd_{0}(\E)\to 0 \]
	we see that $c_{1}(\sEnd_{0}(\E))=c_{1}(\sEnd{\E})=0$.
	Therefore
	\[ \operatorname{ch}(\sEnd_{0}(\E)\ot \w)=3+c_{1}(\sEnd_{0}(\E))+3c_{1}(\w)=3+3c_{1}(\w). \]
	Multiplying the two expressions we obtain
	\[ \operatorname{ch}(\sEnd_{0}(\E)\ot \w)\operatorname{td}(\w\dual)=3+\frac{3}{2}c_{1}(\w), \]
	whose codimension $1$ part has degree $3g-3\geqslant 3$.
	So Hirzebruch--Riemann--Roch tells us that
	\[ h^{0}(M,\sEnd_{0}(\E)\ot \w)-h^{1}(M,\sEnd_{0}(\E)\ot \w)=3g-3\geqslant 3, \]
	which implies that $h^{0}(\P(E),p^{*}(\w\ot \Lambda^{2}\E)(2))=h^{0}(M,\sEnd_{0}(\E)\ot \w)\geqslant 3$.

	Thus, our linear system $|p^{*}(\w\ot \Lambda^{2}\E)(2)|$ has dimension $2$.
	If it does not have any fixed component, then Bertini's theorem \cite[Theorem 7.19]{iit82} and the discussion above imply that a general $\Phi\in H^{0}(M,\sEnd_{0}(\E)\ot \w)$ leaves no line bundle $L\subseteq E$ invariant, i.e.~$E\in \bfA$.

	Let us see what happens if it does have some fixed divisor.
	By definition, a fixed divisor corresponds to a non-zero global section $s_{0}\in H^{0}(\P(E),\M_{1})$ for some invertible sheaf $\M_{1}$ such that there exists another invertible sheaf $\M_{2}$ with $\M_{1}\ot \M_{2}\cong p^{*}(\w\ot\Lambda^{2}\E)(2)$.
	Being a fixed divisor translates into saying that every global section $s\in H^{0}(\P(E),p^{*}(\w\ot \Lambda^{2}\E)(2))$ can be written as $ts_{0}$ for some $t\in H^{0}(\P(E),\M_{2})$.
	In particular, both $H^{0}(\P(E),\M_{1})$ and $H^{0}(\P(E),\M_{2})$ have to be non-zero.
	By \cite[Exercise II.7.4]{har77} we can write $\M_{i}\cong p^{*}\L_{i}(l_{i})$ with $l_{1}+l_{2}=2$.
	In fact, we must have $0\leqslant l_{i}\leqslant 2$, because using again the projection formula we have
	\[ H^{0}(\P(E),p^{*}\L_{i}(l_{i}))\cong H^{0}(M,\L_{i}\ot p_{*}\O(l_{i})) \]
	and $p_{*}\O(l)=0$ for all $l<0$ \cite[Exercise III.8.4]{har77}.
	So we only have the following three possibilities:
	\begin{enumerate}[label=\alph*)]
	    \item $l_{1}=0$;
	    \item $l_{1}=1$;
	    \item $l_{1}=2$.
	\end{enumerate}
	Let us start with case $a)$.
	Let $p^{*}s\in H^{0}(\P(E),p^{*}\L)\cong H^{0}(M,\L)$ be a global section corresponding to the fixed component of our linear system.
	Dividing all global sections by $s$ and by $p^{*}s$ respectively we obtain the following commutative diagram:
	\begin{center}
	    \begin{tikzcd}
		H^{0}(M,\End_{0}(E)\ot K)\arrow{r}{\cong}\arrow[swap]{d}{/s}[swap]{\cong} & H^{0}(\P(E),p^{*}(K\ot \Lambda^{2}E)(2))\arrow{d}{\cong}[swap]{/p^{*}s} \\
		H^{0}(M,\End_{0}(E)\ot K\ot L\dual)\arrow{r}{\cong} & H^{0}(\P(E),p^{*}(L\dual \ot K\ot \Lambda^{2}E)(2))
	    \end{tikzcd}
	\end{center}
	By definition, the new linear system does not have any fixed divisors and has the same dimension.
	Hence we can apply Bertini to conclude that a generic $\Phi'\in H^{0}(M,\End_{0}(E)\ot K\ot L\dual)$ does not have invariant line bundles, which in this case are defined as line bundles $N\subseteq E$ such that $\Phi'(N)\subseteq N\ot K\ot L\dual$.
	But a line bundle $N\subseteq E$ is $\Phi'$-invariant if and only if it is $s\Phi'$-invariant, so we have $E\in \bfA$ in this case.

	We move on to case $b)$.
	Assume that the fixed divisor corresponds to a non-zero global section $s\in H^{0}(\P(E),p^{*}\L(1))$.
	This corresponds to a non-zero morphism $\E\to \L$, whose kernel $\sN\subseteq \E$ must then be an invertible sheaf because it is torsion-free of rank $1$ over the algebraic curve $M$.
    \end{proof}
\end{prop}

\bibliographystyle{alpha}
\bibliography{main}
\vspace{0.3cm}

\end{document}
