\documentclass[A4paper, 12pt, british, reqno]{amsart}

\input{preamble} % Content described in README.md

% Math operators
\DeclareMathOperator{\Hom}{Hom}
\DeclareMathOperator{\End}{End}
\DeclareMathOperator{\Sym}{Sym}
\DeclareMathOperator{\rk}{rk}
\DeclareMathOperator{\id}{id}
\DeclareMathOperator{\ad}{ad}
\DeclareMathOperator{\GL}{GL}
\let\div\relax
\DeclareMathOperator{\div}{div}
\DeclareMathOperator{\SO}{SO}
\let\hom\relax
\DeclareMathOperator{\hom}{hom}

% Other commands
\newcommand{\ot}{\otimes}
\newcommand{\op}{\oplus}
\newcommand{\dual}{^{*}}
\newcommand{\act}{\rotatebox[origin=c]{-120}{$\circlearrowright$}}
\newcommand{\Diff}{\mathbf{Diff}}
\newcommand{\Grp}{\mathbf{Grp}}

\begin{document}

\maketitle

\begin{abstract}
    In this talk we introduce the stability condition for Higgs bundles and prove the Hitchin--Kobayashi correspondence.
    The main result is \cite[Theorem 4.3]{hit87a}.
    Relevant literature is \cite[\S 3 and \S 4]{hit87a} and \cite[\S 2 and \S 3]{wen14}.
    Maybe we will also use \cite{wen16} every now and then.

    This talk is related to Tanuj's talk on \textit{Stable vector bundles}, for which the main reference is \cite{kob87}.
    Therefore we will also use \cite{kob87} as a default reference for generalities on complex vector bundles.
\end{abstract}

\tableofcontents

\section*{Notation and conventions}

We usually follow the notation of \cite{hit87a}:
\begin{itemize}
    \item $M$: compact Riemann surface of genus $g$.
    \item $O\to M$: trivial line bundle.
    \item $K\to M$: canonical line bundle.
    \item More generally, $O_{X}$ resp.~$K_{X}$ denote the trivial resp.~canonical line bundle on a complex manifold $X$.
    \item For a vector bundle $V\to M$ we denote $\mu(V):=\deg{V}/\rk{V}$.
\end{itemize}

\section{Stability}

%[explain that the self-duality equations were originally talking about some principal $G$-bundle on space time $\R^{4}$, but after a series of simplifications \cite[\S 1]{hit87a} we ended up talking about rank 2 complex vector bundles on a compact Riemann surface]

%[picture of moduli space of holomorphic rank 2 vector bundles]
%
%[Upshot: a rank 2 holomorphic vector bundle doesn't quite determine a solution to the self-duality equation, but almost! Vector bundle + tangent vector in its moduli space does.]
%
%[Use fact from moduli space + Serre duality to translate this tangent vector into a Higgs field]
%
%From now on let $M$ be a compcat Riemann surface (of genus $g\geqslant 2$) and $K\to M$ its canonical line bundle.

\begin{defn}[Higgs bundle]
    A \textit{Higgs bundle} on $M$ is a pair $(V,\Phi)$, where $V\to M$ is a rank $2$ vector bundle and $\Phi$ is a global section of $\End{V}\ot K$, called a \textit{Higgs field} on $V$.
\end{defn}

\begin{rem}
    Using the canonical isomorphisms
    \[ H^{0}(M,\End{V}\ot K)\cong \Hom(O,V\dual \ot V\ot K)\cong \Hom(V,V\ot K) \]
    we may identify $\Phi$ with a morphism
    \[ \Phi\colon V\to V\ot K. \]
\end{rem}

%\begin{rem}
%    We are mainly interested in the subbundle of traceless endomorphisms, denoted $\End_{0}{V}$.
%\end{rem}

\begin{exa}
    Assume $g\geqslant 2$.
    Then $\deg{K}=2g-2>0$, so we can find a line bundle $K^{\frac{1}{2}}$ such that $K^{\frac{1}{2}}\ot K^{\frac{1}{2}}\cong K$.
    Let $V=K^{\frac{1}{2}}\op K^{-\frac{1}{2}}$, where $K^{-\frac{1}{2}}=(K^{\frac{1}{2}})^{-1}$.
    We consider the Higgs field $\Phi_{w}\colon K^{\frac{1}{2}}\op K^{-\frac{1}{2}}\to (K^{\frac{1}{2}}\op K^{-\frac{1}{2}})\ot K$ given by a matrix
    \[
	\begin{pmatrix}
	    0 & w \\
	    1 & 0
	\end{pmatrix},
    \]
    where $w\in \Hom(K^{-\frac{1}{2}},K^{\frac{1}{2}}\ot K)\cong H^{0}(M,K^{2})$ can be regarded as a quadratic differential.
\end{exa}

\begin{defn}[Stability]
A Higgs bundle $(V,\Phi)$ is said to be \textit{stable} if for every $\Phi$-invariant\footnote{Meaning that $\Phi(L)\subseteq L\ot K$.} line bundle $L\subseteq V$ we have $\mu(L)<\mu(V)$.
\end{defn}

\begin{rem}
    $(V,0)$ is stable if and only if $V$ is stable in the usual sense.
\end{rem}

\begin{exe}
    There are no stable Higgs bundles on $\P^{1}$.
    [Hint: Grothedieck's theorem allows us to write $\Phi$ as a matrix. What can we say about each entry?]
    [Solution in \cite{hit87a}]
\end{exe}

\begin{exa}
    Assume $g\geqslant 2$ and consier $V=K^{\frac{1}{2}}\op K^{-\frac{1}{2}}$ again.
    Then $\Phi_{0}$ is stable, because $K^{-\frac{1}{2}}$ is the only $\Phi_{0}$-invariant line bundle and
    \[ \deg{K^{-\frac{1}{2}}}=1-g<0=\frac{\deg{V}}{2}. \]
\end{exa}

\begin{prop}
    Assume $g\geqslant 2$ and let $V\to M$ be a rank $2$ vector bundle.
Then there exists Higgs field $\Phi$ on $V$ such that $(V,\Phi)$ is stable if and only if there exists a dense Zariski open subset $U\subseteq H^{0}(M,\End{V}\ot K)$ such that all $\Phi'\in U$ have the property that no line bundle $L\subseteq V$ is $\Phi'$-invariant.
    \begin{proof}
	Let $p\colon P(V)\to M$ be the projectivisation of our rank 2 vector bundle, which is the $\P^{1}$-bundle obtained by replacing each fibre $V_{x}$ by its projectivisation $(V_{x}\setminus \{0\})/\C^{\times}$.
	Let $S\subseteq p^{*}V$ be the tautological line bundle on $P(V)$, whose fibre over a point $[v]\in p^{-1}(x)$ is given by the line $\{ \lambda v \mid \lambda \in \C \}\subseteq V_{x}$.
	Let $H:=S\dual$ be its dual, which fits into a short exact sequence
	\[ 0\to Q\dual \to p^{*}V\dual \to H \to 0. \]
	Let $U\subseteq M$ be an open subset trivialising $V$.
	Then the quotient map $p^{*}V\dual \to H$ in the previous short exact sequence induces an isomorphism
	\[ H^{0}(p^{-1}(U),p^{*}V\dual)\cong H^{0}(p^{-1}(U),H), \]
	so the pushforward of the sheaf of sections of $p^{*}V\dual$ is isomorphic to the pushforward of the sheaf of sections of $H$.
	Since $p$ has connected fibres we have $p_{*}\O_{P(V)}\cong \O_{M}$, so applying the projection formula \cite[Exercise II.5.1.d]{har77} we deduce that the pushforward of the sheaf of sections of $H$ is isomorphic to the sheaf of sections of $V\dual$.
	Abusing slightly the notation we will express this as $V\dual \cong p_{*}H$, and similarly we have $\Sym^{2}V\dual\cong p_{*}H^{2}$.

	Let $x\in M$.
	Then every endomorphism $A\in \End(V_{x})$ defines a quadratic map $V_{x}\to \Lambda^{2}V_{x}$ sending $v$ to $Av\wedge v$.
	Such a quadratic map can be naturally regarded as a degree $2$ homogeneous polynomial on the coordinates coordinates of $v$ with coefficients in $\Lambda^{2}V_{x}$.
	Hence we have a vector bundle morphism $\End(V)\to S^{2}V\ot \Lambda^{2}V$, which vanishes precisely along the trivial line subbundle of $\End(V)$ consisting over each fibre of scalar multiples of the identity.
	Sending $A\mapsto A-\frac{\operatorname{tr}(A)}{2}\id_{V_{x}}$ on each fibre allows us to identify $\End_{0}(V)$ as the quotient of $\End(V)$ by this trivial line subbundle, so we obtain an injective morphism $\End_{0}(V)\to S^{2}V\ot \Lambda^{2}V$.
	Counting dimensions we see that we have in fact an isomorphism of vector bundles $\End_{0}(V)\cong S^{2}V\ot \Lambda^{2}V$, and therefore
	\[ \End_{0}(V)\ot K\cong p_{*}H^{2}K\ot \Lambda^{2}V. \]
	Using again that $p_{*}p^{*}(-)\cong (-)$ for vector bundles we have a $\C$-linear isomorphism
	\[ s\colon H^{0}(M,\End_{0}(V)\ot K)\cong H^{0}(P(V),H^{2}p^{*}(K\ot \Lambda^{2}V)). \]

	Let now $\Phi$ be a traceless Higgs field on $V$, and assume it is non-zero.
	By construction, a non-zero vector $v\in V$ is an eigenvector of the twisted endomorphism over the corresponding fibre if and only if the section $s(\Phi)$ vanishes at the point $[v]\in P(V)$, i.e.~if and only if $[v]$ is in the divisor of zeros of the section $s(\Phi)$, which we denote $\div(s(\Phi))$.
	Let $L\subseteq V$ be a $\Phi$-invariant subbundle, which defines a section of $p\colon P(V)\to M$ by functoriality of projectivisation on injective morphisms of vector bundles:
	\begin{center}
	    \begin{tikzcd}
		P(L)\arrow[equal]{d}\arrow{r}{\sigma} & P(L)\arrow{dl}{p} \\
		M &
	    \end{tikzcd}
	\end{center}
	Being $\Phi$-invariant means precisely that $\sigma(M)\subseteq \div(s(\Phi))$.
	But then any non-zero $v\in L$ is a non-zero eigenvector of the endomorphism over the corresponding fibre.
	Since $\Phi$ was non-zero, we can assume that the corresponding eigenvalue is non-zero as well.
	Since $\Phi$ is traceless, the other eigenvalue must be different, and there must be some non-zero eigenvector outside of $L$, call it $u\in V$.
	Since $u$ is a non-zero eigenvector, $[u]\in \div(s(\Phi))$.
	And since $u\not\in L$, $[u]\not\in \sigma(M)$.
	Therefore $\sigma(M)$ is a proper irreducible component of the divisor $\div(s(\Phi))$.
	In conclusion: if $\div(s(\Phi))$ is irreducible, then no line bundle $L\subseteq V$ is $\Phi$-invariant.

	Our next goal is to show that $\div(s(\Phi))$ is irreducible for generic $\Phi$.
	We will show that the linear system $H^{2}p^{*}(K\ot\Lambda^{2}V)$ has dimension $2$ and has no fixed component, and then irreducibility follows from Bertini's theorem \cite[Theorem 3.3.1]{laz04a}.

	The dimension of the linear system is one less than the dimension of the vector space of global secitons.
	Using the previous isomorphism it suffices to gain control over the dimension of the global sections of $\End_{0}{V}\ot K$ on $M$, for which we can apply Hirzebruch--Riemann--Roch \cite[Theorem A.4.1]{har77}.
	From \cite[Example A.4.1.1]{har77} we get
	\[ \operatorname{td}(K\dual)=1-\frac{c_{1}(K)}{2}. \]
	Using the short exact sequence used earlier
	\[ 0\to O\to \End(V) \to \End_{0}(V)\to 0 \]
	we see that $c_{1}(\End_{0}{V})=c_{1}(\End{V})=0$.
	Therefore
	\[ \operatorname{ch}(\End_{0}{V}\ot K)=3+c_{1}(\End_{0}{V})+3c_{1}(K)=3+3c_{1}(K). \]
	Multiplying the two expressions we obtain
	\[ \operatorname{ch}(\End_{0}{V}\ot K)\operatorname{td}(K\dual)=3+\frac{3}{2}c_{1}(K), \]
	whose codimension $1$ part has degree $3g-3\geqslant 3$.
	So Hirzebruch--Riemann--Roch tells us that
	\[ h^{0}(M,\End_{0}{V}\ot K)-h^{1}(M,\End_{0}{V}\ot K)=3g-3\geqslant 3, \]
	which implies that $h^{0}(P(V),H^{2}p^{*}(K\ot \Lambda^{2}V))=h^{0}(M,\End_{0}{V}\ot K)\geqslant 3$, as we wanted to show.
    \end{proof}
\end{prop}

\bibliographystyle{alpha}
\bibliography{main}
\vspace{0.3cm}

\end{document}
