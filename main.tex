\documentclass[A4paper, 12pt, british, reqno]{amsart}

\input{preamble} % Content described in README.md

% Math operators
\DeclareMathOperator{\Hom}{Hom}
\DeclareMathOperator{\End}{End}
\DeclareMathOperator{\Sym}{Sym}
\DeclareMathOperator{\rk}{rk}
\DeclareMathOperator{\id}{id}
\DeclareMathOperator{\ad}{ad}
\DeclareMathOperator{\GL}{GL}
\DeclareMathOperator{\SO}{SO}
\let\hom\relax
\DeclareMathOperator{\hom}{hom}

% Other commands
\newcommand{\ot}{\otimes}
\newcommand{\op}{\oplus}
\newcommand{\dual}{^{\vee}}
\newcommand{\act}{\rotatebox[origin=c]{-120}{$\circlearrowright$}}
\newcommand{\Diff}{\mathbf{Diff}}
\newcommand{\Grp}{\mathbf{Grp}}

\begin{document}

\maketitle

\begin{abstract}
    In this talk we introduce the stability condition for Higgs bundles and prove the Hitchin--Kobayashi correspondence.
    The main result is \cite[Theorem 4.3]{hit87}.
    Relevant literature is \cite[\S 3 and \S 4]{hit87} and \cite[\S 2 and \S 3]{wen14}.
    Maybe we will also use \cite{wen16} every now and then.

    This talk is related to Tanuj's talk on \textit{Stable vector bundles}, for which the main reference is \cite{kob87}.
    Therefore we will also use \cite{kob87} as a main reference for generalities on complex vector bundles.
\end{abstract}

\tableofcontents

\section*{Notation and conventions}

All morphisms, vector bundles and sections are holomorphic unless otherwise specified.

\section{Stability}

[explain that the self-duality equations were originally talking about some principal $G$-bundle on space time $\R^{4}$, but after a series of simplifications \cite[\S 1]{hit87} we ended up talking about rank 2 complex vector bundles on a compact Riemann surface]

[picture of moduli space of holomorphic rank 2 vector bundles]

[Upshot: a rank 2 holomorphic vector bundle doesn't quite determine a solution to the self-duality equation, but almost! Vector bundle + tangent vector in its moduli space does.]

[Use fact from moduli space + Serre duality to translate this tangent vector into a Higgs field]

From now on let $M$ be a compcat Riemann surface (of genus $g\geqslant 2$) and $K\to M$ its canonical line bundle.

\begin{defn}[Higgs bundles]
A \textit{Higgs bundle} on $M$ is a pair $(V,\Phi)$, where $V\to M$ is a vector bundle on $M$ and $\Phi\colon M\to K\ot \End(V)$ is a global section of $K\ot \End{V}$.
    We call $\Phi$ a \textit{Higgs field} on $V$.
\end{defn}

\begin{rem}
    $\Phi$ can be regarded as morphism
    \[ \Phi\colon V\to V\ot K \]
    because $K\ot \End{V}\cong K\ot V\ot V\dual\cong \Hom(V,V\ot K)$.
\end{rem}

\begin{rem}
    We are mainly interested in rank $2$ vector bundles and traceless endomorphisms.
\end{rem}

\begin{exa}
    Let $M$ have genus $g>1$.
    Then $\deg{K}=2g-2>0$, so we can find a line bundle $K^{\frac{1}{2}}$ such that
    \[ K^{\frac{1}{2}}\ot K^{\frac{1}{2}}\cong K. \]
    Let $V=K^{\frac{1}{2}}\op K^{-\frac{1}{2}}$, where $K^{-\frac{1}{2}}=(K^{\frac{1}{2}})^{-1}$.
    We consider the Higgs field $\Phi_{w}\colon K^{\frac{1}{2}}\op K^{-\frac{1}{2}}\to (K^{\frac{1}{2}}\op K^{-\frac{1}{2}})\ot K$ given by a matrix
    \[
	\begin{pmatrix}
	    0 & w \\
	    1 & 0
	\end{pmatrix},
    \]
    where $w\in \Hom(K^{-\frac{1}{2}},K^{\frac{1}{2}}\ot K)\cong H^{0}(M,K^{2})$ can be regarded as a quadratic differential.

    Therefore we have a family $\{(V,\Phi_{w})\}_{w}$ of Higgs bundles on $M$ parametrised by quadratic differential forms $w$ on $M$.
\end{exa}

Let us denote the \textit{slope} of a vector bundle $V\to M$ as
\[ \mu(V):=\frac{\deg{V}}{\rk{V}}. \]

\begin{defn}[Stability of Higgs bundles]
    Let $(V,\Phi)$ be a Higgs bundle on $M$.
    A vector subbundle $W\subseteq V$ is said to be \textit{$\Phi$-invariant} if $\Phi(W)\subseteq W\ot K$.
    We say that $(V,\Phi)$ is...
    \begin{enumerate}[label=\roman*)]
	\item \textit{stable} if for every $\Phi$-invariant subbundle $0\subsetneq W\subsetneq V$ we have $\mu(W) <\mu(V)$.
	\item \textit{semi-stable} if for every $\Phi$-invariant subbundle $0\subsetneq W\subsetneq V$ we have $\mu(W)\leqslant \mu(V)$.
    \end{enumerate}
\end{defn}

\begin{rem}
    The Higgs bundle $(V,0)$ is (semi-)stable if and only if $V$ is (semi-)stable in the usual sense, because every vector subbundle is $0$-invariant.
\end{rem}

\begin{exa}[Omit during the talk]
    On $\P^{1}$, we can write every rank $2$ vector bundle as
    \[ V\cong O(a)\op O(b) \]
    for some integers $a\geqslant b$, where $O(-1)$ is the tautological line bundle on $\P^{1}$ and $O(a):=O(-1)^{-a}$.
    Let $\Phi\in \Hom(V,V\ot K)$ be given by the matrix
    \[
	\begin{pmatrix}
	    0 & \theta_{1} \\
	    \theta_{2} & 0
	\end{pmatrix}
    \]
    with $\theta_{1}\in \Hom(O(b),O(a)\ot K)$ and $\theta_{2}\in \Hom(O(a),O(b)\ot K)$.
    Since $K\cong O(-2)$, we can also regard
    \[ \theta_{1}\in H^{0}(\P^{1},O(a-b-2)) \text{ and } \theta_{2}\in H^{0}(\P^{1},O(b-a-2)). \]
    Since $a\geqslant b$, the line bundle $O(b-a-2)$ does not have any global sections.
    Hence $\theta_{2}=0$ and $O(a)\subseteq V$ is $\Phi$-invariant.
    But
    \[ \deg{O(a)}=a\geqslant \frac{a+b}{2}=\frac{\deg{V}}{2}, \]
    so $(V,\Phi)$ cannot be a stable Higgs bundle.
\end{exa}

\begin{exa}
    Let $M$ have genus $g\geqslant 2$ and consier $V=K^{\frac{1}{2}}\op K^{-\frac{1}{2}}$ again.
    Then $\Phi_{0}$ is stable, because $K^{-\frac{1}{2}}$ is the only $\Phi_{0}$-invariant line bundle and
    \[ \deg{K^{-\frac{1}{2}}}=1-g<0=\frac{\deg{V}}{2}. \]
\end{exa}

\begin{prop}
    Let $M$ have genus $g\geqslant 2$.
    a rank $2$ vector bundle $V$ occurs in a stable Higgs bundle $(V,\Phi)$ if and only if one of the following holds:
    \begin{enumerate}[label=\roman*)]
	\item $V$ is stable;
	\item $V$ is semi-stable and $g>2$;
	\item $V$ is semi-stgable, $g=2$ and $V\cong U\ot L$, where $U$ is either a direct sum of line bundles or an extension of the form
	    \[ 0\to O\to U\to O\to 0, \]
	    where $O=M\times \C\to M$ is the trivial line bundle.
	\item $V$ is not semi-stable and $h^{0}(M,L_{V}^{-2}\ot K\ot \det{V})>1$, where $L_{V}\subseteq V$ is the unique rank 1 subbundle with $\mu(L_{V})\geqslant \mu(V)$;
	\item $V$ is a direct sum of line bundles of the form
	    \[ V\cong L_{V}\op (L_{V}^{-1}\ot \det{V}) \]
	    and $h^{0}(M,L_{V}^{-2}\ot K\ot \det{V})=1$, where $L_{V}\subseteq V$ is again the unique rank 1 subbundle with $\mu(L_{V})\geqslant \mu(V)$.
    \end{enumerate}
    \begin{proof}
	Let $p\colon \P(V)\to M$ be the projectivisation of our rank 2 vector bundle, which is the $\P^{1}$-bundle obtained by replacing each fibre $V_{x}$ of $V\to M$ by its projectivisation $\P(V_{x}):=(V_{x}\setminus \{0\})/\C^{\times}$.
	Let $S\subseteq p^{*}V$ be the tautological line bundle on $\P(V)$, whose fibre over a point $([v],x)\in p^{-1}(x)$ is given by the line
	\[ \{ (\lambda v, x)\in V_{x}\times \{x\}\mid \lambda \in \C \}. \]
	Let $H:=S\dual$ be its dual, which fits into a short exact sequence
	\[ 0\to Q\dual \to p^{*}V\dual \to H \to 0. \]
	Let $\varnothing\neq U\subseteq M$ be an open subset trivialising $V$.
	Then the quotient map $p^{*}V\dual \to H$ induces an isomorphism
	\[ H^{0}(p^{-1}(U),p^{*}V\dual)\cong H^{0}(p^{-1}(U),H), \]
	so the pushforward of the sheaf of sections of $p^{*}V\dual$ is isomorphic to the pushforward of the sheaf of sections of $H$.
	Since $p$ has connected fibres we have $p_{*}\O_{\P(V)}\cong \O_{M}$, so applying the projection formula \cite[Exercise II.5.1.d]{har77} we deduce that the pushforward of the sheaf of sections of $H$ is isomorphic to the sheaf of sections of $V\dual$.
	Abusing slightly the notation, from now on we will write these kind of statements as
	\[ V\dual \cong p_{*}H. \]
	With the same argument we can show that $\Sym^{2}V\dual \cong p_{*}H^{2}$.
    \end{proof}
\end{prop}

\bibliographystyle{alpha}
\bibliography{main}
\vspace{0.3cm}

\end{document}
